\section{模型}\label{about-keras-models}
\subsection{关于Keras模型}
在 Keras 中有两类主要的模型:\hyperref[sequential-api]{Sequential 顺序模型} 和 \hyperref[model-api]{使用函数式 API 的 Model 类模型}。

这些模型有许多共同的方法和属性:

\begin{itemize}
\tightlist
\item \texttt{model.layers}是包含模型网络层的展平列表。
\item \texttt{model.inputs}是模型输入张量的列表。
\item \texttt{model.outputs}是模型输出张量的列表。
\item
  \texttt{model.summary()}: 打印出模型概述信息。 它是
  \hyperref[print-summary]{utils.print\_summary} 的简捷调用。
\item
  \texttt{model.get\_config()}:
  返回包含模型配置信息的字典。通过以下代码,就可以根据这些配置信息重新实例化模型:
\end{itemize}

\begin{Shaded}
\begin{Highlighting}[]
\NormalTok{config }\OperatorTok{=} \NormalTok{model.get_config()}
\NormalTok{model }\OperatorTok{=} \NormalTok{Model.from_config(config)}
\CommentTok{# or, for Sequential:}
\NormalTok{model }\OperatorTok{=} \NormalTok{Sequential.from_config(config)}
\end{Highlighting}
\end{Shaded}

\begin{itemize}
\tightlist
\item
  \texttt{model.get\_weights()}: 返回模型权重的张量列表,类型为 Numpy
  array。
\item
  \texttt{model.set\_weights(weights)}: 从 Nympy array
  中为模型设置权重。列表中的数组必须与 \texttt{get\_weights()}
  返回的权重具有相同的尺寸。
\item
  \texttt{model.to\_json()}: 以 JSON
  字符串的形式返回模型的表示。请注意,该表示不包括权重,只包含结构。你可以通过以下代码,从
  JSON 字符串中重新实例化相同的模型(带有重新初始化的权重):
\end{itemize}

\begin{Shaded}
\begin{Highlighting}[]
\ImportTok{from} \NormalTok{keras.models }\ImportTok{import} \NormalTok{model_from_json}

\NormalTok{json_string }\OperatorTok{=} \NormalTok{model.to_json()}
\NormalTok{model }\OperatorTok{=} \NormalTok{model_from_json(json_string)}
\end{Highlighting}
\end{Shaded}

\begin{itemize}
\tightlist
\item
  \texttt{model.to\_yaml()}: 以 YAML
  字符串的形式返回模型的表示。请注意,该表示不包括权重,只包含结构。你可以通过以下代码,从
  YAML 字符串中重新实例化相同的模型(带有重新初始化的权重):
\end{itemize}

\begin{Shaded}
\begin{Highlighting}[]
\ImportTok{from} \NormalTok{keras.models }\ImportTok{import} \NormalTok{model_from_yaml}

\NormalTok{yaml_string }\OperatorTok{=} \NormalTok{model.to_yaml()}
\NormalTok{model }\OperatorTok{=} \NormalTok{model_from_yaml(yaml_string)}
\end{Highlighting}
\end{Shaded}

\begin{itemize}
\tightlist
\item
  \texttt{model.save\_weights(filepath)}: 将模型权重存储为 HDF5 文件。
\item
  \texttt{model.load\_weights(filepath,\ by\_name=False)}: 从 HDF5
  文件(由 \texttt{save\_weights}
  创建)中加载权重。默认情况下,模型的结构应该是不变的。
  如果想将权重载入不同的模型(部分层相同), 设置 \texttt{by\_name=True}
  来载入那些名字相同的层的权重。
\end{itemize}

注意:另请参阅\hyperref[how-can-i-install-HDF5-or-h5py-to-save-my-models-in-Keras]{如何安装HDF5 或 h5py 以保存 Keras 模型},在常见问题中了解如何安装 \texttt{h5py}
的说明。


\textbf{Model 子类}

除了这两类模型之外,你还可以通过继承 \texttt{Model} 类并在 \texttt{call}
方法中实现你自己的前向传播,以创建你自己的完全定制化的模型,(\texttt{Model} 子类 API 引入于 Keras 2.2.0)。

这里是一个用 \texttt{Model} 子类写的简单的多层感知器的例子:

\begin{Shaded}
\begin{Highlighting}[]
\ImportTok{import}\NormalTok{ keras}

\KeywordTok{class}\NormalTok{ SimpleMLP(keras.Model):}

    \KeywordTok{def} \FunctionTok{__init__}\NormalTok{(}\VariableTok{self}\NormalTok{, use_bn}\OperatorTok{=}\VariableTok{False}\NormalTok{, use_dp}\OperatorTok{=}\VariableTok{False}\NormalTok{, num_classes}\OperatorTok{=}\DecValTok{10}\NormalTok{):}
        \BuiltInTok{super}\NormalTok{(SimpleMLP, }\VariableTok{self}\NormalTok{).}\FunctionTok{__init__}\NormalTok{(name}\OperatorTok{=}\StringTok{'mlp'}\NormalTok{)}
        \VariableTok{self}\NormalTok{.use_bn }\OperatorTok{=}\NormalTok{ use_bn}
        \VariableTok{self}\NormalTok{.use_dp }\OperatorTok{=}\NormalTok{ use_dp}
        \VariableTok{self}\NormalTok{.num_classes }\OperatorTok{=}\NormalTok{ num_classes}

        \VariableTok{self}\NormalTok{.dense1 }\OperatorTok{=}\NormalTok{ keras.layers.Dense(}\DecValTok{32}\NormalTok{, activation}\OperatorTok{=}\StringTok{'relu'}\NormalTok{)}
        \VariableTok{self}\NormalTok{.dense2 }\OperatorTok{=}\NormalTok{ keras.layers.Dense(num_classes, activation}\OperatorTok{=}\StringTok{'softmax'}\NormalTok{)}
        \ControlFlowTok{if} \VariableTok{self}\NormalTok{.use_dp:}
            \VariableTok{self}\NormalTok{.dp }\OperatorTok{=}\NormalTok{ keras.layers.Dropout(}\FloatTok{0.5}\NormalTok{)}
        \ControlFlowTok{if} \VariableTok{self}\NormalTok{.use_bn:}
            \VariableTok{self}\NormalTok{.bn }\OperatorTok{=}\NormalTok{ keras.layers.BatchNormalization(axis}\OperatorTok{=-}\DecValTok{1}\NormalTok{)}

    \KeywordTok{def}\NormalTok{ call(}\VariableTok{self}\NormalTok{, inputs):}
\NormalTok{        x }\OperatorTok{=} \VariableTok{self}\NormalTok{.dense1(inputs)}
        \ControlFlowTok{if} \VariableTok{self}\NormalTok{.use_dp:}
\NormalTok{            x }\OperatorTok{=} \VariableTok{self}\NormalTok{.dp(x)}
        \ControlFlowTok{if} \VariableTok{self}\NormalTok{.use_bn:}
\NormalTok{            x }\OperatorTok{=} \VariableTok{self}\NormalTok{.bn(x)}
        \ControlFlowTok{return} \VariableTok{self}\NormalTok{.dense2(x)}

\NormalTok{model }\OperatorTok{=}\NormalTok{ SimpleMLP()}
\NormalTok{model.}\BuiltInTok{compile}\NormalTok{(...)}
\NormalTok{model.fit(...)}
\end{Highlighting}
\end{Shaded}

网络层定义在 \texttt{\_\_init\_\_(self,\ ...)} 中,前向传播在
\texttt{call(self,\ inputs)} 中指定。在 \texttt{call}
中,你可以指定自定义的损失函数,通过调用
\texttt{self.add\_loss(loss\_tensor)} (就像你在自定义层中一样)。

在子类模型中,模型的拓扑结构是由 Python
代码定义的(而不是网络层的静态图)。这意味着该模型的拓扑结构不能被检查或序列化。因此,以下方法和属性不适用于子类模型:

\begin{itemize}
\tightlist
\item
  \texttt{model.inputs} 和 \texttt{model.outputs}。
\item
  \texttt{model.to\_yaml()} 和 \texttt{model.to\_json()}。
\item
  \texttt{model.get\_config()} 和 \texttt{model.save()}。
\end{itemize}

关键点:为每个任务使用正确的 API。\texttt{Model} 子类化 API 可以为实现复杂模型提供更大的灵活性,但它需要付出代价(比如缺失的特性):它更冗长,更复杂,并且有更多的用户错误机会。如果可能的话,尽可能使用函数式API,这对用户更友好。
