\section{工具}


\subsection{ CustomObjectScope {\href{https://github.com/keras-team/keras/blob/master/keras/utils/generic_utils.py\#L20}{{[}source{]}}}}\label{customobjectscope}

\begin{Shaded}
\begin{Highlighting}[]
\NormalTok{keras.utils.CustomObjectScope()}
\end{Highlighting}
\end{Shaded}

提供一个无法转义的\texttt{\_GLOBAL\_CUSTOM\_OBJECTS} 范围。

\texttt{with} 语句中的代码将能够通过名称访问自定义对象。
对全局自定义对象的更改会在封闭的 \texttt{with} 语句中持续存在。
在\texttt{with}语句结束时, 全局自定义对象将恢复到 \texttt{with}
语句开始时的状态。

\textbf{例子}

考虑自定义对象 \texttt{MyObject} (例如一个类):

\begin{Shaded}
\begin{Highlighting}[]
\ControlFlowTok{with} \NormalTok{CustomObjectScope(\{}\StringTok{'MyObject'}\NormalTok{:MyObject\}):}
    \NormalTok{layer }\OperatorTok{=} \NormalTok{Dense(..., kernel_regularizer}\OperatorTok{=}\StringTok{'MyObject'}\NormalTok{)}
	\CommentTok{# 保存,加载等操作将按这个名称来识别自定义对象}
\end{Highlighting}
\end{Shaded}




\subsection{ HDF5Matrix {\href{https://github.com/keras-team/keras/blob/master/keras/utils/io_utils.py\#L16}{{[}source{]}}}}\label{HDF5Matrix}

\begin{Shaded}
\begin{Highlighting}[]
\NormalTok{keras.utils.HDF5Matrix(datapath, dataset, start}\OperatorTok{=}\DecValTok{0}\NormalTok{, end}\OperatorTok{=}\VariableTok{None}\NormalTok{, normalizer}\OperatorTok{=}\VariableTok{None}\NormalTok{)}
\end{Highlighting}
\end{Shaded}

使用 HDF5 数据集表示,而不是 Numpy 数组。

\textbf{例子}

\begin{Shaded}
\begin{Highlighting}[]
\NormalTok{x_data }\OperatorTok{=} \NormalTok{HDF5Matrix(}\StringTok{'input/file.hdf5'}\NormalTok{, }\StringTok{'data'}\NormalTok{)}
\NormalTok{model.predict(x_data)}
\end{Highlighting}
\end{Shaded}

提供 \texttt{start} 和 \texttt{end} 将允许使用数据集的一个切片。

你还可以给出标准化函数(或 lambda)(可选)。
这将在检索到的每一个数据切片上调用它。

\textbf{参数}

\begin{itemize}
\tightlist
\item
  \textbf{datapath}: 字符串,HDF5 文件路径。
\item
  \textbf{dataset}: 字符串,datapath指定的文件中的 HDF5 数据集名称。
\item
  \textbf{start}: 整数,所需的指定数据集的切片的开始位置。
\item
  \textbf{end}: 整数,所需的指定数据集的切片的结束位置。
\item
  \textbf{normalizer}: 在检索数据时调用的函数。
\end{itemize}

\textbf{返回}

一个类数组的 HDF5 数据集。




\subsection{ Sequence {\href{https://github.com/keras-team/keras/blob/master/keras/utils/data_utils.py\#L303}{{[}source{]}}}}

\begin{Shaded}
\begin{Highlighting}[]
\NormalTok{keras.utils.Sequence()}
\end{Highlighting}
\end{Shaded}

用于拟合数据序列的基类,例如一个数据集。

每一个 \texttt{Sequence} 必须实现 \texttt{\_\_getitem\_\_} 和
\texttt{\_\_len\_\_} 方法。 如果你想在迭代之间修改你的数据集,你可以实现
\texttt{on\_epoch\_end}。 \texttt{\_\_getitem\_\_}
方法应该范围一个完整的批次。

\textbf{注意}

\texttt{Sequence}
是进行多进程处理的更安全的方法。这种结构保证网络在每个时期每个样本只训练一次,这与生成器不同。

\textbf{例子}

\begin{Shaded}
\begin{Highlighting}[]
\ImportTok{from} \NormalTok{skimage.io }\ImportTok{import} \NormalTok{imread}
\ImportTok{from} \NormalTok{skimage.transform }\ImportTok{import} \NormalTok{resize}
\ImportTok{import} \NormalTok{numpy }\ImportTok{as} \NormalTok{np}
\ImportTok{import} \NormalTok{math}

\CommentTok{# 这里,`x_set` 是图像的路径列表}
\CommentTok{# 以及 `y_set` 是对应的类别}

\KeywordTok{class} \NormalTok{CIFAR10Sequence(Sequence):}

    \KeywordTok{def} \FunctionTok{__init__}\NormalTok{(}\VariableTok{self}\NormalTok{, x_set, y_set, batch_size):}
        \VariableTok{self}\NormalTok{.x, }\VariableTok{self}\NormalTok{.y }\OperatorTok{=} \NormalTok{x_set, y_set}
        \VariableTok{self}\NormalTok{.batch_size }\OperatorTok{=} \NormalTok{batch_size}

    \KeywordTok{def} \FunctionTok{__len__}\NormalTok{(}\VariableTok{self}\NormalTok{):}
        \ControlFlowTok{return} \NormalTok{math.ceil(}\BuiltInTok{len}\NormalTok{(}\VariableTok{self}\NormalTok{.x) }\OperatorTok{/} \VariableTok{self}\NormalTok{.batch_size)}

    \KeywordTok{def} \FunctionTok{__getitem__}\NormalTok{(}\VariableTok{self}\NormalTok{, idx):}
        \NormalTok{batch_x }\OperatorTok{=} \VariableTok{self}\NormalTok{.x[idx }\OperatorTok{*} \VariableTok{self}\NormalTok{.batch_size:(idx }\OperatorTok{+} \DecValTok{1}\NormalTok{) }\OperatorTok{*} \VariableTok{self}\NormalTok{.batch_size]}
        \NormalTok{batch_y }\OperatorTok{=} \VariableTok{self}\NormalTok{.y[idx }\OperatorTok{*} \VariableTok{self}\NormalTok{.batch_size:(idx }\OperatorTok{+} \DecValTok{1}\NormalTok{) }\OperatorTok{*} \VariableTok{self}\NormalTok{.batch_size]}

        \ControlFlowTok{return} \NormalTok{np.array([}
            \NormalTok{resize(imread(file_name), (}\DecValTok{200}\NormalTok{, }\DecValTok{200}\NormalTok{))}
               \ControlFlowTok{for} \NormalTok{file_name }\OperatorTok{in} \NormalTok{batch_x]), np.array(batch_y)}
\end{Highlighting}
\end{Shaded}



\subsection{to\_categorical}\label{toux5fcategorical}

\begin{Shaded}
\begin{Highlighting}[]
\NormalTok{keras.utils.to_categorical(y, num_classes}\OperatorTok{=}\VariableTok{None}\NormalTok{)}
\end{Highlighting}
\end{Shaded}

将类向量(整数)转换为二进制类矩阵。

例如。用于 categorical\_crossentropy。

\textbf{参数}

\begin{itemize}
\tightlist
\item
  \textbf{y}: 需要转换成矩阵的类矢量 (从 0 到 num\_classes 的整数)。
\item
  \textbf{num\_classes}: 总类别数。
\end{itemize}

\textbf{返回}

输入的二进制矩阵表示。



\subsection{normalize}\label{normalize}

\begin{Shaded}
\begin{Highlighting}[]
\NormalTok{keras.utils.normalize(x, axis}\OperatorTok{=-}\DecValTok{1}\NormalTok{, order}\OperatorTok{=}\DecValTok{2}\NormalTok{)}
\end{Highlighting}
\end{Shaded}

标准化一个 Numpy 数组。

\textbf{参数}

\begin{itemize}
\tightlist
\item
  \textbf{x}: 需要标准化的 Numpy 数组。
\item
  \textbf{axis}: 需要标准化的轴。
\item
  \textbf{order}: 标准化顺序(例如,2 表示 L2 规范化)。
\end{itemize}

\textbf{Returns}

数组的标准化副本。



\subsection{get\_file}\label{get-file}

\begin{Shaded}
\begin{Highlighting}[]
\NormalTok{keras.utils.get_file(fname, origin, untar}\OperatorTok{=}\VariableTok{False}\NormalTok{, md5_hash}\OperatorTok{=}\VariableTok{None}\NormalTok{, file_hash}\OperatorTok{=}\VariableTok{None}, \\
\hspace{3cm}\NormalTok{cache_subdir}\OperatorTok{=}\StringTok{'datasets'}\NormalTok{, hash_algorithm}\OperatorTok{=}\StringTok{'auto'}\NormalTok{, extract}\OperatorTok{=}\VariableTok{False}, \\
\hspace{3cm}\NormalTok{archive_format}\OperatorTok{=}\StringTok{'auto'}\NormalTok{, cache_dir}\OperatorTok{=}\VariableTok{None}\NormalTok{)}
\end{Highlighting}
\end{Shaded}

从一个 URL 下载文件,如果它不存在缓存中。

默认情况下,URL \texttt{origin}处的文件 被下载到缓存目录
\texttt{\textasciitilde{}/.keras} 中, 放在缓存子目录 \texttt{datasets}中,并命名为
\texttt{fname}。 文件 \texttt{example.txt} 的最终位置为
\texttt{\textasciitilde{}/.keras/datasets/example.txt}。

tar, tar.gz, tar.bz, 以及 zip 格式的文件也可以被解压。
传递一个哈希值将在下载后校验文件。 命令行程序 \texttt{shasum} 和
\texttt{sha256sum} 可以计算哈希。

\textbf{参数}

\begin{itemize}
\tightlist
\item
  \textbf{fname}: 文件名。如果指定了绝对路径
  \texttt{/path/to/file.txt}, 那么文件将会保存到那个路径。
\item
  \textbf{origin}: 文件的原始 URL。
\item
  \textbf{untar}: 由于使用 'extract' 而已被弃用。
  布尔值,是否需要解压文件。
\item
  \textbf{md5\_hash}: 由于使用 'file\_hash' 而已被弃用。 用于校验的 md5
  哈希值。
\item
  \textbf{file\_hash}: 下载后的文件的期望哈希字符串。 支持 sha256 和 md5
  两个哈希算法。
\item
  \textbf{cache\_subdir}: 在 Keras 缓存目录下的保存文件的子目录。
  如果指定了绝对路径 \texttt{/path/to/folder},则文件将被保存在该位置。
\item
  \textbf{hash\_algorithm}: 选择哈希算法来校验文件。 可选的有 'md5',
  'sha256', 以及 'auto'。 默认的 'auto' 将自动检测所使用的哈希算法。
\item
  \textbf{extract}: True 的话会尝试将解压缩存档文件,如tar或zip。
\item
  \textbf{archive\_format}: 尝试提取文件的存档格式。 可选的有 'auto',
  'tar', 'zip', 以及 None。 'tar' 包含 tar, tar.gz, 和 tar.bz 文件。
  默认 'auto' 为 {[}'tar', 'zip'{]}。 None 或
  空列表将返回未找到任何匹配。 ke xu az z'auto', 'tar', 'zip', and None.
\item
  \textbf{cache\_dir}: 存储缓存文件的位置,为 None 时默认为
  \hyperref[where-is-the-keras-configuration-file-stored]{Keras
  目录}.
\end{itemize}

\textbf{返回}

下载的文件的路径。



\subsection{print\_summary}\label{print-summary}

\begin{Shaded}
\begin{Highlighting}[]
\NormalTok{keras.utils.print_summary(model, line_length}\OperatorTok{=}\VariableTok{None}\NormalTok{, positions}\OperatorTok{=}\VariableTok{None}\NormalTok{, print_fn}\OperatorTok{=}\VariableTok{None}\NormalTok{)}
\end{Highlighting}
\end{Shaded}

打印模型概况。

\textbf{参数}

\begin{itemize}
\tightlist
\item
  \textbf{model}: Keras 模型实例。
\item
  \textbf{line\_length}: 打印的每行的总长度
  (例如,设置此项以使其显示适应不同的终端窗口大小)。
\item
  \textbf{positions}: 每行中日志元素的相对或绝对位置。
  如果未提供,默认为 \texttt{{[}.33,\ .55,\ .67,\ 1.{]}}。
\item
  \textbf{print\_fn}: 需要使用的打印函数。 它将在每一行概述时调用。
  您可以将其设置为自定义函数以捕获字符串概述。 默认为 \texttt{print}
  (打印到标准输出)。
\end{itemize}



\subsection{plot\_model}\label{plotux5fmodel}

\begin{Shaded}
\begin{Highlighting}[]
\NormalTok{keras.utils.plot_model(model, to_file}\OperatorTok{=}\StringTok{'model.png'}\NormalTok{, show_shapes}\OperatorTok{=}\VariableTok{False}, \\
\hspace{3cm}\NormalTok{show_layer_names}\OperatorTok{=}\VariableTok{True}\NormalTok{, rankdir}\OperatorTok{=}\StringTok{'TB'}\NormalTok{)}
\end{Highlighting}
\end{Shaded}

将 Keras 模型转换为 dot 格式并保存到文件中。

\textbf{参数}

\begin{itemize}
\tightlist
\item
  \textbf{model}: 一个 Keras 模型实例。
\item
  \textbf{to\_file}: 绘制图像的文件名。
\item
  \textbf{show\_shapes}: 是否显示尺寸信息。
\item
  \textbf{show\_layer\_names}: 是否显示层的名称。
\item
  \textbf{rankdir}: 传递给 PyDot 的 \texttt{rankdir} 参数,
  一个指定绘图格式的字符串: 'TB' 创建一个垂直绘图; 'LR'
  创建一个水平绘图。
\end{itemize}



\subsection{multi\_gpu\_model}\label{multi-gpu-model}

\begin{Shaded}
\begin{Highlighting}[]
\NormalTok{keras.utils.multi_gpu_model(model, gpus}\OperatorTok{=}\VariableTok{None},
\NormalTok{cpu_merge}\OperatorTok{=}\VariableTok{True},
\NormalTok{cpu_relocation}\OperatorTok{=}\VariableTok{False}\NormalTok{)}
\end{Highlighting}
\end{Shaded}

将模型复制到不同的 GPU 上。

具体来说,该功能实现了单机多 GPU 数据并行性。 它的工作原理如下:

\begin{itemize}
\tightlist
\item
  将模型的输入分成多个子批次。
\item
  在每个子批次上应用模型副本。 每个模型副本都在专用 GPU 上执行。
\item
  将结果(在 CPU 上)连接成一个大批量。
\end{itemize}

例如, 如果你的 \texttt{batch\_size} 是 64,且你使用 \texttt{gpus=2},
那么我们将把输入分为两个 32 个样本的子批次, 在 1 个 GPU 上处理 1
个子批次,然后返回完整批次的 64 个处理过的样本。

这实现了多达 8 个 GPU 的准线性加速。

此功能目前仅适用于 TensorFlow 后端。

\textbf{参数}

\begin{itemize}
\tightlist
\item
  \textbf{model}: 一个 Keras 模型实例。为了避免OOM错误,该模型可以建立在
  CPU 上, 详见下面的使用样例。
\item
  \textbf{gpus}: 整数 \textgreater{}= 2 或整数列表,创建模型副本的 GPU
  数量, 或 GPU ID 的列表。
\item
  \textbf{cpu\_merge}: 一个布尔值,用于标识是否强制合并 CPU
  范围内的模型权重。
\item
  \textbf{cpu\_relocation}: 一个布尔值,用来确定是否在 CPU
  的范围内创建模型的权重。如果模型没有在任何一个设备范围内定义,您仍然可以通过激活这个选项来拯救它。
\end{itemize}

\textbf{返回}

一个 Keras \texttt{Model} 实例,它可以像初始 \texttt{model}
参数一样使用,但它将工作负载分布在多个 GPU 上。




\textbf{例子}

\begin{Shaded}
\begin{Highlighting}[]
\ImportTok{import} \NormalTok{tensorflow }\ImportTok{as} \NormalTok{tf}
\ImportTok{from} \NormalTok{keras.applications }\ImportTok{import} \NormalTok{Xception}
\ImportTok{from} \NormalTok{keras.utils }\ImportTok{import} \NormalTok{multi_gpu_model}
\ImportTok{import} \NormalTok{numpy }\ImportTok{as} \NormalTok{np}

\NormalTok{num_samples }\OperatorTok{=} \DecValTok{1000}
\NormalTok{height }\OperatorTok{=} \DecValTok{224}
\NormalTok{width }\OperatorTok{=} \DecValTok{224}
\NormalTok{num_classes }\OperatorTok{=} \DecValTok{1000}

\CommentTok{# 实例化基础模型(或「模板」模型)。}
\CommentTok{# 我们建议在 CPU 设备范围内执行此操作,}
\CommentTok{# 以便让模型的权值托管在 CPU 内存上。}
\CommentTok{# 否则,他们可能会最终托管在 GPU 上,}
\CommentTok{# 这会使权值共享变得复杂。}
\ControlFlowTok{with} \NormalTok{tf.device(}\StringTok{'/cpu:0'}\NormalTok{):}
    \NormalTok{model }\OperatorTok{=} \NormalTok{Xception(weights}\OperatorTok{=}\VariableTok{None}\NormalTok{,}
                     \NormalTok{input_shape}\OperatorTok{=}\NormalTok{(height, width, }\DecValTok{3}\NormalTok{),}
                     \NormalTok{classes}\OperatorTok{=}\NormalTok{num_classes)}

\CommentTok{# 将模型复制到 8 个 GPU 上。}
\CommentTok{# 这假定你的机器有 8 个可用的 GPU。}
\NormalTok{parallel_model }\OperatorTok{=} \NormalTok{multi_gpu_model(model, gpus}\OperatorTok{=}\DecValTok{8}\NormalTok{)}
\NormalTok{parallel_model.}\BuiltInTok{compile}\NormalTok{(loss}\OperatorTok{=}\StringTok{'categorical_crossentropy'}\NormalTok{,}
                       \NormalTok{optimizer}\OperatorTok{=}\StringTok{'rmsprop'}\NormalTok{)}

\CommentTok{# 生成虚拟数据。}
\NormalTok{x }\OperatorTok{=} \NormalTok{np.random.random((num_samples, height, width, }\DecValTok{3}\NormalTok{))}
\NormalTok{y }\OperatorTok{=} \NormalTok{np.random.random((num_samples, num_classes))}

\CommentTok{# 这个 `fit` 调用将分布在 8 个 GPU 上。}
\CommentTok{# 由于 批大小为 256,每个 GPU 将处理 32 个样本。}
\NormalTok{parallel_model.fit(x, y, epochs}\OperatorTok{=}\DecValTok{20}\NormalTok{, batch_size}\OperatorTok{=}\DecValTok{256}\NormalTok{)}

\CommentTok{# 通过模板模型(共享相同的权值)保存模型:}
\NormalTok{model.save(}\StringTok{'my_model.h5'}\NormalTok{)}
\end{Highlighting}
\end{Shaded}



\textbf{关于模型保存}

要保存多 GPU 模型,请通过模板模型(传递给 multi\_gpu\_model 的参数)调用
.save(fname) 或 .save\_weights(fname) 以进行存储,而不是通过
multi\_gpu\_model 返回的模型。

\newpage
