\subsection{函数式 API 指引}
\subsubsection{开始使用 Keras 函数式 API}\label{functional-api-guide}

Keras 函数式 API
是定义复杂模型(如多输出模型、有向无环图,或具有共享层的模型)的方法。

这部分文档假设你已经对 Sequential 顺序模型比较熟悉。

让我们先从一些简单的例子开始。


\subsubsection{例一:全连接网络}\label{ux4f8bux4e00ux5168ux8fdeux63a5ux7f51ux7edc}

\texttt{Sequential}
模型可能是实现这种网络的一个更好选择,但这个例子能够帮助我们进行一些简单的理解。

\begin{itemize}
\tightlist
\item
  网络层的实例是可调用的,它以张量为参数,并且返回一个张量
\item
  输入和输出均为张量,它们都可以用来定义一个模型(\texttt{Model})
\item
  这样的模型同 Keras 的 \texttt{Sequential} 模型一样,都可以被训练
\end{itemize}

\begin{Shaded}
\begin{Highlighting}[]
\ImportTok{from} \NormalTok{keras.layers }\ImportTok{import} \NormalTok{Input, Dense}
\ImportTok{from} \NormalTok{keras.models }\ImportTok{import} \NormalTok{Model}

\CommentTok{# 这部分返回一个张量}
\NormalTok{inputs }\OperatorTok{=} \NormalTok{Input(shape}\OperatorTok{=}\NormalTok{(}\DecValTok{784}\NormalTok{,))}

\CommentTok{# 层的实例是可调用的,它以张量为参数,并且返回一个张量}
\NormalTok{x }\OperatorTok{=} \NormalTok{Dense(}\DecValTok{64}\NormalTok{, activation}\OperatorTok{=}\StringTok{'relu'}\NormalTok{)(inputs)}
\NormalTok{x }\OperatorTok{=} \NormalTok{Dense(}\DecValTok{64}\NormalTok{, activation}\OperatorTok{=}\StringTok{'relu'}\NormalTok{)(x)}
\NormalTok{predictions }\OperatorTok{=} \NormalTok{Dense(}\DecValTok{10}\NormalTok{, activation}\OperatorTok{=}\StringTok{'softmax'}\NormalTok{)(x)}

\CommentTok{# 这部分创建了一个包含输入层和三个全连接层的模型}
\NormalTok{model }\OperatorTok{=} \NormalTok{Model(inputs}\OperatorTok{=}\NormalTok{inputs, outputs}\OperatorTok{=}\NormalTok{predictions)}
\NormalTok{model.}\BuiltInTok{compile}\NormalTok{(optimizer}\OperatorTok{=}\StringTok{'rmsprop'}\NormalTok{,}
              \NormalTok{loss}\OperatorTok{=}\StringTok{'categorical_crossentropy'}\NormalTok{,}
              \NormalTok{metrics}\OperatorTok{=}\NormalTok{[}\StringTok{'accuracy'}\NormalTok{])}
\NormalTok{model.fit(data, labels)  }\CommentTok{# 开始训练}
\end{Highlighting}
\end{Shaded}


\subsubsection{所有的模型都可调用,就像网络层一样}\label{ux6240ux6709ux7684ux6a21ux578bux90fdux53efux8c03ux7528ux5c31ux50cfux7f51ux7edcux5c42ux4e00ux6837}

利用函数式
API,可以轻易地重用训练好的模型:可以将任何模型看作是一个层,然后通过传递一个张量来调用它。注意,在调用模型时,您不仅重用模型的\emph{结构},还重用了它的权重。

\begin{Shaded}
\begin{Highlighting}[]
\NormalTok{x }\OperatorTok{=} \NormalTok{Input(shape}\OperatorTok{=}\NormalTok{(}\DecValTok{784}\NormalTok{,))}
\CommentTok{# 这是可行的,并且返回上面定义的 10-way softmax。}
\NormalTok{y }\OperatorTok{=} \NormalTok{model(x)}
\end{Highlighting}
\end{Shaded}

这种方式能允许我们快速创建可以处理\emph{序列输入}的模型。只需一行代码,你就将图像分类模型转换为视频分类模型。

\begin{Shaded}
\begin{Highlighting}[]
\ImportTok{from} \NormalTok{keras.layers }\ImportTok{import} \NormalTok{TimeDistributed}

\CommentTok{# 输入张量是 20 个时间步的序列,每一个时间为一个 784 维的向量}
\NormalTok{input_sequences }\OperatorTok{=} \NormalTok{Input(shape}\OperatorTok{=}\NormalTok{(}\DecValTok{20}\NormalTok{, }\DecValTok{784}\NormalTok{))}

\CommentTok{# 这部分将我们之前定义的模型应用于输入序列中的每个时间步。}
\CommentTok{# 之前定义的模型的输出是一个 10-way softmax,}
\CommentTok{# 因而下面的层的输出将是维度为 10 的 20 个向量的序列。}
\NormalTok{processed_sequences }\OperatorTok{=} \NormalTok{TimeDistributed(model)(input_sequences)}
\end{Highlighting}
\end{Shaded}


\subsubsection{多输入多输出模型}\label{ux591aux8f93ux5165ux591aux8f93ux51faux6a21ux578b}

以下是函数式 API 的一个很好的例子:具有多个输入和输出的模型。函数式 API
使处理大量交织的数据流变得容易。

来考虑下面的模型。我们试图预测 Twitter
上的一条新闻标题有多少转发和点赞数。模型的主要输入将是新闻标题本身,即一系列词语,但是为了增添趣味,我们的模型还添加了其他的辅助输入来接收额外的数据,例如新闻标题的发布的时间等。
该模型也将通过两个损失函数进行监督学习。较早地在模型中使用主损失函数,是深度学习模型的一个良好正则方法。

模型结构如下图所示:

让我们用函数式 API 来实现它。

主要输入接收新闻标题本身,即一个整数序列(每个整数编码一个词)。
这些整数在 1 到 10,000 之间(10,000 个词的词汇表),且序列长度为 100
个词。

\begin{Shaded}
\begin{Highlighting}[]
\ImportTok{from} \NormalTok{keras.layers }\ImportTok{import} \NormalTok{Input, Embedding, LSTM, Dense}
\ImportTok{from} \NormalTok{keras.models }\ImportTok{import} \NormalTok{Model}

\CommentTok{# 标题输入:接收一个含有 100 个整数的序列,每个整数在 1 到 10000 之间。}
\CommentTok{# 注意我们可以通过传递一个 `name` 参数来命名任何层。}
\NormalTok{main_input }\OperatorTok{=} \NormalTok{Input(shape}\OperatorTok{=}\NormalTok{(}\DecValTok{100}\NormalTok{,), dtype}\OperatorTok{=}\StringTok{'int32'}\NormalTok{, name}\OperatorTok{=}\StringTok{'main_input'}\NormalTok{)}

\CommentTok{# Embedding 层将输入序列编码为一个稠密向量的序列,每个向量维度为 512。}
\NormalTok{x }\OperatorTok{=} \NormalTok{Embedding(output_dim}\OperatorTok{=}\DecValTok{512}\NormalTok{, input_dim}\OperatorTok{=}\DecValTok{10000}\NormalTok{, input_length}\OperatorTok{=}\DecValTok{100}\NormalTok{)(main_input)}

\CommentTok{# LSTM 层把向量序列转换成单个向量,它包含整个序列的上下文信息}
\NormalTok{lstm_out }\OperatorTok{=} \NormalTok{LSTM(}\DecValTok{32}\NormalTok{)(x)}
\end{Highlighting}
\end{Shaded}

在这里,我们插入辅助损失,使得即使在模型主损失很高的情况下,LSTM 层和
Embedding 层都能被平稳地训练。

\begin{Shaded}
\begin{Highlighting}[]
\NormalTok{auxiliary_output }\OperatorTok{=} \NormalTok{Dense(}\DecValTok{1}\NormalTok{, activation}\OperatorTok{=}\StringTok{'sigmoid'}\NormalTok{, name}\OperatorTok{=}\StringTok{'aux_output'}\NormalTok{)(lstm_out)}
\end{Highlighting}
\end{Shaded}

此时,我们将辅助输入数据与 LSTM 层的输出连接起来,输入到模型中:

\begin{Shaded}
\begin{Highlighting}[]
\NormalTok{auxiliary_input }\OperatorTok{=} \NormalTok{Input(shape}\OperatorTok{=}\NormalTok{(}\DecValTok{5}\NormalTok{,), name}\OperatorTok{=}\StringTok{'aux_input'}\NormalTok{)}
\NormalTok{x }\OperatorTok{=} \NormalTok{keras.layers.concatenate([lstm_out, auxiliary_input])}

\CommentTok{# 堆叠多个全连接网络层}
\NormalTok{x }\OperatorTok{=} \NormalTok{Dense(}\DecValTok{64}\NormalTok{, activation}\OperatorTok{=}\StringTok{'relu'}\NormalTok{)(x)}
\NormalTok{x }\OperatorTok{=} \NormalTok{Dense(}\DecValTok{64}\NormalTok{, activation}\OperatorTok{=}\StringTok{'relu'}\NormalTok{)(x)}
\NormalTok{x }\OperatorTok{=} \NormalTok{Dense(}\DecValTok{64}\NormalTok{, activation}\OperatorTok{=}\StringTok{'relu'}\NormalTok{)(x)}

\CommentTok{# 最后添加主要的逻辑回归层}
\NormalTok{main_output }\OperatorTok{=} \NormalTok{Dense(}\DecValTok{1}\NormalTok{, activation}\OperatorTok{=}\StringTok{'sigmoid'}\NormalTok{, name}\OperatorTok{=}\StringTok{'main_output'}\NormalTok{)(x)}
\end{Highlighting}
\end{Shaded}

然后定义一个具有两个输入和两个输出的模型:

\begin{Shaded}
\begin{Highlighting}[]
\NormalTok{model }\OperatorTok{=} \NormalTok{Model(inputs}\OperatorTok{=}\NormalTok{[main_input, auxiliary_input], outputs}\OperatorTok{=}\NormalTok{[main_output, auxiliary_output])}
\end{Highlighting}
\end{Shaded}

现在编译模型,并给辅助损失分配一个 0.2
的权重。如果要为不同的输出指定不同的 \texttt{loss\_weights} 或
\texttt{loss},可以使用列表或字典。 在这里,我们给 \texttt{loss}
参数传递单个损失函数,这个损失将用于所有的输出。

\begin{Shaded}
\begin{Highlighting}[]
\NormalTok{model.}\BuiltInTok{compile}\NormalTok{(optimizer}\OperatorTok{=}\StringTok{'rmsprop'}\NormalTok{, loss}\OperatorTok{=}\StringTok{'binary_crossentropy'}\NormalTok{,}
              \NormalTok{loss_weights}\OperatorTok{=}\NormalTok{[}\DecValTok{1}\NormalTok{., }\FloatTok{0.2}\NormalTok{])}
\end{Highlighting}
\end{Shaded}

我们可以通过传递输入数组和目标数组的列表来训练模型:

\begin{Shaded}
\begin{Highlighting}[]
\NormalTok{model.fit([headline_data, additional_data], [labels, labels],}
          \NormalTok{epochs}\OperatorTok{=}\DecValTok{50}\NormalTok{, batch_size}\OperatorTok{=}\DecValTok{32}\NormalTok{)}
\end{Highlighting}
\end{Shaded}

由于输入和输出均被命名了(在定义时传递了一个 \texttt{name}
参数),我们也可以通过以下方式编译模型:

\begin{Shaded}
\begin{Highlighting}[]
\NormalTok{model.}\BuiltInTok{compile}\NormalTok{(optimizer}\OperatorTok{=}\StringTok{'rmsprop'}\NormalTok{,}
     \NormalTok{loss}\OperatorTok{=}\NormalTok{\{}\StringTok{'main_output'}\NormalTok{: }\StringTok{'binary_crossentropy'}\NormalTok{, }\StringTok{'aux_output'}\NormalTok{: }\StringTok{'binary_crossentropy'}\NormalTok{\},}
    \NormalTok{loss_weights}\OperatorTok{=}\NormalTok{\{}\StringTok{'main_output'}\NormalTok{: }\DecValTok{1}\NormalTok{., }\StringTok{'aux_output'}\NormalTok{: }\FloatTok{0.2}\NormalTok{\})}

\CommentTok{# 然后使用以下方式训练:}
\NormalTok{model.fit(\{}\StringTok{'main_input'}\NormalTok{: headline_data, }\StringTok{'aux_input'}\NormalTok{: additional_data\},}
    \NormalTok{\{}\StringTok{'main_output'}\NormalTok{: labels, }\StringTok{'aux_output'}\NormalTok{: labels\},}
     \NormalTok{epochs}\OperatorTok{=}\DecValTok{50}\NormalTok{, batch_size}\OperatorTok{=}\DecValTok{32}\NormalTok{)}
\end{Highlighting}
\end{Shaded}


\subsubsection{共享网络层}\label{the-concept-of-layer-node}

函数式 API 的另一个用途是使用共享网络层的模型。我们来看看共享层。

来考虑推特推文数据集。我们想要建立一个模型来分辨两条推文是否来自同一个人(例如,通过推文的相似性来对用户进行比较)。

实现这个目标的一种方法是建立一个模型,将两条推文编码成两个向量,连接向量,然后添加逻辑回归层;这将输出两条推文来自同一作者的概率。模型将接收一对对正负表示的推特数据。

由于这个问题是对称的,编码第一条推文的机制应该被完全重用来编码第二条推文。这里我们使用一个共享的
LSTM 层来编码推文。

让我们使用函数式 API 来构建它。首先我们将一条推特转换为一个尺寸为
\texttt{(140,\ 256)} 的矩阵,即每条推特 140 字符,每个字符为 256 维的
one-hot 编码 (取 256 个常用字符)。

\begin{Shaded}
\begin{Highlighting}[]
\ImportTok{import} \NormalTok{keras}
\ImportTok{from} \NormalTok{keras.layers }\ImportTok{import} \NormalTok{Input, LSTM, Dense}
\ImportTok{from} \NormalTok{keras.models }\ImportTok{import} \NormalTok{Model}

\NormalTok{tweet_a }\OperatorTok{=} \NormalTok{Input(shape}\OperatorTok{=}\NormalTok{(}\DecValTok{140}\NormalTok{, }\DecValTok{256}\NormalTok{))}
\NormalTok{tweet_b }\OperatorTok{=} \NormalTok{Input(shape}\OperatorTok{=}\NormalTok{(}\DecValTok{140}\NormalTok{, }\DecValTok{256}\NormalTok{))}
\end{Highlighting}
\end{Shaded}

要在不同的输入上共享同一个层,只需实例化该层一次,然后根据需要传入你想要的输入即可:

\begin{Shaded}
\begin{Highlighting}[]
\CommentTok{# 这一层可以输入一个矩阵,并返回一个 64 维的向量}
\NormalTok{shared_lstm }\OperatorTok{=} \NormalTok{LSTM(}\DecValTok{64}\NormalTok{)}

\CommentTok{# 当我们重用相同的图层实例多次,图层的权重也会被重用 (它其实就是同一层)}
\NormalTok{encoded_a }\OperatorTok{=} \NormalTok{shared_lstm(tweet_a)}
\NormalTok{encoded_b }\OperatorTok{=} \NormalTok{shared_lstm(tweet_b)}

\CommentTok{# 然后再连接两个向量:}
\NormalTok{merged_vector }\OperatorTok{=} \NormalTok{keras.layers.concatenate([encoded_a, encoded_b], axis}\OperatorTok{=-}\DecValTok{1}\NormalTok{)}

\CommentTok{# 再在上面添加一个逻辑回归层}
\NormalTok{predictions }\OperatorTok{=} \NormalTok{Dense(}\DecValTok{1}\NormalTok{, activation}\OperatorTok{=}\StringTok{'sigmoid'}\NormalTok{)(merged_vector)}

\CommentTok{# 定义一个连接推特输入和预测的可训练的模型}
\NormalTok{model }\OperatorTok{=} \NormalTok{Model(inputs}\OperatorTok{=}\NormalTok{[tweet_a, tweet_b], outputs}\OperatorTok{=}\NormalTok{predictions)}

\NormalTok{model.}\BuiltInTok{compile}\NormalTok{(optimizer}\OperatorTok{=}\StringTok{'rmsprop'}\NormalTok{,}
              \NormalTok{loss}\OperatorTok{=}\StringTok{'binary_crossentropy'}\NormalTok{,}
              \NormalTok{metrics}\OperatorTok{=}\NormalTok{[}\StringTok{'accuracy'}\NormalTok{])}
\NormalTok{model.fit([data_a, data_b], labels, epochs}\OperatorTok{=}\DecValTok{10}\NormalTok{)}
\end{Highlighting}
\end{Shaded}

让我们暂停一会,看看如何读取共享层的输出或输出尺寸。


\subsubsection{层「节点」的概念}\label{ux5c42ux8282ux70b9ux7684ux6982ux5ff5}

每当你在某个输入上调用一个层时,都将创建一个新的张量(层的输出),并且为该层添加一个「节点」,将输入张量连接到输出张量。当多次调用同一个图层时,该图层将拥有多个节点索引
(0, 1, 2...)。

在之前版本的 Keras 中,可以通过 \texttt{layer.get\_output()}
来获得层实例的输出张量,或者通过 \texttt{layer.output\_shape}
来获取其输出形状。现在你依然可以这么做(除了 \texttt{get\_output()}
已经被 \texttt{output} 属性替代)。但是如果一个层与多个输入连接呢?

只要一个层只连接到一个输入,就不会有困惑,\texttt{.output}
会返回层的唯一输出:

\begin{Shaded}
\begin{Highlighting}[]
\NormalTok{a }\OperatorTok{=} \NormalTok{Input(shape}\OperatorTok{=}\NormalTok{(}\DecValTok{140}\NormalTok{, }\DecValTok{256}\NormalTok{))}

\NormalTok{lstm }\OperatorTok{=} \NormalTok{LSTM(}\DecValTok{32}\NormalTok{)}
\NormalTok{encoded_a }\OperatorTok{=} \NormalTok{lstm(a)}

\ControlFlowTok{assert} \NormalTok{lstm.output }\OperatorTok{==} \NormalTok{encoded_a}
\end{Highlighting}
\end{Shaded}

但是如果该层有多个输入,那就会出现问题:

\begin{Shaded}
\begin{Highlighting}[]
\NormalTok{a }\OperatorTok{=} \NormalTok{Input(shape}\OperatorTok{=}\NormalTok{(}\DecValTok{140}\NormalTok{, }\DecValTok{256}\NormalTok{))}
\NormalTok{b }\OperatorTok{=} \NormalTok{Input(shape}\OperatorTok{=}\NormalTok{(}\DecValTok{140}\NormalTok{, }\DecValTok{256}\NormalTok{))}

\NormalTok{lstm }\OperatorTok{=} \NormalTok{LSTM(}\DecValTok{32}\NormalTok{)}
\NormalTok{encoded_a }\OperatorTok{=} \NormalTok{lstm(a)}
\NormalTok{encoded_b }\OperatorTok{=} \NormalTok{lstm(b)}

\NormalTok{lstm.output}
\end{Highlighting}
\end{Shaded}

\begin{verbatim}
>> AttributeError: Layer lstm_1 has multiple inbound nodes,
hence the notion of "layer output" is ill-defined.
Use `get_output_at(node_index)` instead.
\end{verbatim}

好吧,通过下面的方法可以解决:

\begin{Shaded}
\begin{Highlighting}[]
\ControlFlowTok{assert} \NormalTok{lstm.get_output_at(}\DecValTok{0}\NormalTok{) }\OperatorTok{==} \NormalTok{encoded_a}
\ControlFlowTok{assert} \NormalTok{lstm.get_output_at(}\DecValTok{1}\NormalTok{) }\OperatorTok{==} \NormalTok{encoded_b}
\end{Highlighting}
\end{Shaded}

够简单,对吧?

\texttt{input\_shape} 和 \texttt{output\_shape}
这两个属性也是如此:只要该层只有一个节点,或者只要所有节点具有相同的输入/输出尺寸,那么「层输出/输入尺寸」的概念就被很好地定义,并且将由
\texttt{layer.output\_shape} / \texttt{layer.input\_shape}
返回。但是比如说,如果将一个 \texttt{Conv2D} 层先应用于尺寸为
\texttt{(32,32,3)} 的输入,再应用于尺寸为 \texttt{(64,\ 64,\ 3)}
的输入,那么这个层就会有多个输入/输出尺寸,你将不得不通过指定它们所属节点的索引来获取它们:

\begin{Shaded}
\begin{Highlighting}[]
\NormalTok{a }\OperatorTok{=} \NormalTok{Input(shape}\OperatorTok{=}\NormalTok{(}\DecValTok{32}\NormalTok{, }\DecValTok{32}\NormalTok{, }\DecValTok{3}\NormalTok{))}
\NormalTok{b }\OperatorTok{=} \NormalTok{Input(shape}\OperatorTok{=}\NormalTok{(}\DecValTok{64}\NormalTok{, }\DecValTok{64}\NormalTok{, }\DecValTok{3}\NormalTok{))}

\NormalTok{conv }\OperatorTok{=} \NormalTok{Conv2D(}\DecValTok{16}\NormalTok{, (}\DecValTok{3}\NormalTok{, }\DecValTok{3}\NormalTok{), padding}\OperatorTok{=}\StringTok{'same'}\NormalTok{)}
\NormalTok{conved_a }\OperatorTok{=} \NormalTok{conv(a)}

\CommentTok{# 到目前为止只有一个输入,以下可行:}
\ControlFlowTok{assert} \NormalTok{conv.input_shape }\OperatorTok{==} \NormalTok{(}\VariableTok{None}\NormalTok{, }\DecValTok{32}\NormalTok{, }\DecValTok{32}\NormalTok{, }\DecValTok{3}\NormalTok{)}

\NormalTok{conved_b }\OperatorTok{=} \NormalTok{conv(b)}
\CommentTok{# 现在 `.input_shape` 属性不可行,但是这样可以:}
\ControlFlowTok{assert} \NormalTok{conv.get_input_shape_at(}\DecValTok{0}\NormalTok{) }\OperatorTok{==} \NormalTok{(}\VariableTok{None}\NormalTok{, }\DecValTok{32}\NormalTok{, }\DecValTok{32}\NormalTok{, }\DecValTok{3}\NormalTok{)}
\ControlFlowTok{assert} \NormalTok{conv.get_input_shape_at(}\DecValTok{1}\NormalTok{) }\OperatorTok{==} \NormalTok{(}\VariableTok{None}\NormalTok{, }\DecValTok{64}\NormalTok{, }\DecValTok{64}\NormalTok{, }\DecValTok{3}\NormalTok{)}
\end{Highlighting}
\end{Shaded}


\subsubsection{更多的例子}\label{ux66f4ux591aux7684ux4f8bux5b50}

代码示例仍然是起步的最佳方式,所以这里还有更多的例子。

\subsubsubsection{Inception 模型}\label{inception-ux6a21ux578b}

有关 Inception 结构的更多信息,请参阅
\href{http://arxiv.org/abs/1409.4842}{Going Deeper with Convolutions}。

\begin{Shaded}
\begin{Highlighting}[]
\ImportTok{from} \NormalTok{keras.layers }\ImportTok{import} \NormalTok{Conv2D, MaxPooling2D, Input}

\NormalTok{input_img }\OperatorTok{=} \NormalTok{Input(shape}\OperatorTok{=}\NormalTok{(}\DecValTok{256}\NormalTok{, }\DecValTok{256}\NormalTok{, }\DecValTok{3}\NormalTok{))}

\NormalTok{tower_1 }\OperatorTok{=} \NormalTok{Conv2D(}\DecValTok{64}\NormalTok{, (}\DecValTok{1}\NormalTok{, }\DecValTok{1}\NormalTok{), padding}\OperatorTok{=}\StringTok{'same'}\NormalTok{, activation}\OperatorTok{=}\StringTok{'relu'}\NormalTok{)(input_img)}
\NormalTok{tower_1 }\OperatorTok{=} \NormalTok{Conv2D(}\DecValTok{64}\NormalTok{, (}\DecValTok{3}\NormalTok{, }\DecValTok{3}\NormalTok{), padding}\OperatorTok{=}\StringTok{'same'}\NormalTok{, activation}\OperatorTok{=}\StringTok{'relu'}\NormalTok{)(tower_1)}

\NormalTok{tower_2 }\OperatorTok{=} \NormalTok{Conv2D(}\DecValTok{64}\NormalTok{, (}\DecValTok{1}\NormalTok{, }\DecValTok{1}\NormalTok{), padding}\OperatorTok{=}\StringTok{'same'}\NormalTok{, activation}\OperatorTok{=}\StringTok{'relu'}\NormalTok{)(input_img)}
\NormalTok{tower_2 }\OperatorTok{=} \NormalTok{Conv2D(}\DecValTok{64}\NormalTok{, (}\DecValTok{5}\NormalTok{, }\DecValTok{5}\NormalTok{), padding}\OperatorTok{=}\StringTok{'same'}\NormalTok{, activation}\OperatorTok{=}\StringTok{'relu'}\NormalTok{)(tower_2)}

\NormalTok{tower_3 }\OperatorTok{=} \NormalTok{MaxPooling2D((}\DecValTok{3}\NormalTok{, }\DecValTok{3}\NormalTok{), strides}\OperatorTok{=}\NormalTok{(}\DecValTok{1}\NormalTok{, }\DecValTok{1}\NormalTok{), padding}\OperatorTok{=}\StringTok{'same'}\NormalTok{)(input_img)}
\NormalTok{tower_3 }\OperatorTok{=} \NormalTok{Conv2D(}\DecValTok{64}\NormalTok{, (}\DecValTok{1}\NormalTok{, }\DecValTok{1}\NormalTok{), padding}\OperatorTok{=}\StringTok{'same'}\NormalTok{, activation}\OperatorTok{=}\StringTok{'relu'}\NormalTok{)(tower_3)}

\NormalTok{output }\OperatorTok{=} \NormalTok{keras.layers.concatenate([tower_1, tower_2, tower_3], axis}\OperatorTok{=}\DecValTok{1}\NormalTok{)}
\end{Highlighting}
\end{Shaded}

\subsubsubsection{卷积层上的残差连接}\label{ux5377ux79efux5c42ux4e0aux7684ux6b8bux5deeux8fdeux63a5}

有关残差网络 (Residual Network) 的更多信息,请参阅
\href{http://arxiv.org/abs/1512.03385}{Deep Residual Learning for Image
Recognition}。

\begin{Shaded}
\begin{Highlighting}[]
\ImportTok{from} \NormalTok{keras.layers }\ImportTok{import} \NormalTok{Conv2D, Input}

\CommentTok{# 输入张量为 3 通道 256x256 图像}
\NormalTok{x }\OperatorTok{=} \NormalTok{Input(shape}\OperatorTok{=}\NormalTok{(}\DecValTok{256}\NormalTok{, }\DecValTok{256}\NormalTok{, }\DecValTok{3}\NormalTok{))}
\CommentTok{# 3 输出通道(与输入通道相同)的 3x3 卷积核}
\NormalTok{y }\OperatorTok{=} \NormalTok{Conv2D(}\DecValTok{3}\NormalTok{, (}\DecValTok{3}\NormalTok{, }\DecValTok{3}\NormalTok{), padding}\OperatorTok{=}\StringTok{'same'}\NormalTok{)(x)}
\CommentTok{# 返回 x + y}
\NormalTok{z }\OperatorTok{=} \NormalTok{keras.layers.add([x, y])}
\end{Highlighting}
\end{Shaded}

\subsubsubsection{共享视觉模型}\label{ux5171ux4eabux89c6ux89c9ux6a21ux578b}

该模型在两个输入上重复使用同一个图像处理模块,以判断两个 MNIST
数字是否为相同的数字。

\begin{Shaded}
\begin{Highlighting}[]
\ImportTok{from} \NormalTok{keras.layers }\ImportTok{import} \NormalTok{Conv2D, MaxPooling2D, Input, Dense, Flatten}
\ImportTok{from} \NormalTok{keras.models }\ImportTok{import} \NormalTok{Model}

\CommentTok{# 首先,定义视觉模型}
\NormalTok{digit_input }\OperatorTok{=} \NormalTok{Input(shape}\OperatorTok{=}\NormalTok{(}\DecValTok{27}\NormalTok{, }\DecValTok{27}\NormalTok{, }\DecValTok{1}\NormalTok{))}
\NormalTok{x }\OperatorTok{=} \NormalTok{Conv2D(}\DecValTok{64}\NormalTok{, (}\DecValTok{3}\NormalTok{, }\DecValTok{3}\NormalTok{))(digit_input)}
\NormalTok{x }\OperatorTok{=} \NormalTok{Conv2D(}\DecValTok{64}\NormalTok{, (}\DecValTok{3}\NormalTok{, }\DecValTok{3}\NormalTok{))(x)}
\NormalTok{x }\OperatorTok{=} \NormalTok{MaxPooling2D((}\DecValTok{2}\NormalTok{, }\DecValTok{2}\NormalTok{))(x)}
\NormalTok{out }\OperatorTok{=} \NormalTok{Flatten()(x)}

\NormalTok{vision_model }\OperatorTok{=} \NormalTok{Model(digit_input, out)}

\CommentTok{# 然后,定义区分数字的模型}
\NormalTok{digit_a }\OperatorTok{=} \NormalTok{Input(shape}\OperatorTok{=}\NormalTok{(}\DecValTok{27}\NormalTok{, }\DecValTok{27}\NormalTok{, }\DecValTok{1}\NormalTok{))}
\NormalTok{digit_b }\OperatorTok{=} \NormalTok{Input(shape}\OperatorTok{=}\NormalTok{(}\DecValTok{27}\NormalTok{, }\DecValTok{27}\NormalTok{, }\DecValTok{1}\NormalTok{))}

\CommentTok{# 视觉模型将被共享,包括权重和其他所有}
\NormalTok{out_a }\OperatorTok{=} \NormalTok{vision_model(digit_a)}
\NormalTok{out_b }\OperatorTok{=} \NormalTok{vision_model(digit_b)}

\NormalTok{concatenated }\OperatorTok{=} \NormalTok{keras.layers.concatenate([out_a, out_b])}
\NormalTok{out }\OperatorTok{=} \NormalTok{Dense(}\DecValTok{1}\NormalTok{, activation}\OperatorTok{=}\StringTok{'sigmoid'}\NormalTok{)(concatenated)}

\NormalTok{classification_model }\OperatorTok{=} \NormalTok{Model([digit_a, digit_b], out)}
\end{Highlighting}
\end{Shaded}

\subsubsubsection{视觉问答模型}\label{ux89c6ux89c9ux95eeux7b54ux6a21ux578b}

当被问及关于图片的自然语言问题时,该模型可以选择正确的单词作答。

它通过将问题和图像编码成向量,然后连接两者,在上面训练一个逻辑回归,来从词汇表中挑选一个可能的单词作答。

\begin{Shaded}
\begin{Highlighting}[]
\ImportTok{from} \NormalTok{keras.layers }\ImportTok{import} \NormalTok{Conv2D, MaxPooling2D, Flatten}
\ImportTok{from} \NormalTok{keras.layers }\ImportTok{import} \NormalTok{Input, LSTM, Embedding, Dense}
\ImportTok{from} \NormalTok{keras.models }\ImportTok{import} \NormalTok{Model, Sequential}

\CommentTok{# 首先,让我们用 Sequential 来定义一个视觉模型。}
\CommentTok{# 这个模型会把一张图像编码为向量。}
\NormalTok{vision_model }\OperatorTok{=} \NormalTok{Sequential()}
\NormalTok{vision_model.add(Conv2D(}\DecValTok{64}\NormalTok{, (}\DecValTok{3}\NormalTok{, }\DecValTok{3}\NormalTok{), activation}\OperatorTok{=}\StringTok{'relu'}\NormalTok{, padding}\OperatorTok{=}\StringTok{'same'}, \\
\hspace{3cm}\NormalTok{input_shape}\OperatorTok{=}\NormalTok{(}\DecValTok{224}\NormalTok{, }\DecValTok{224}\NormalTok{, }\DecValTok{3}\NormalTok{)))}
\NormalTok{vision_model.add(Conv2D(}\DecValTok{64}\NormalTok{, (}\DecValTok{3}\NormalTok{, }\DecValTok{3}\NormalTok{), activation}\OperatorTok{=}\StringTok{'relu'}\NormalTok{))}
\NormalTok{vision_model.add(MaxPooling2D((}\DecValTok{2}\NormalTok{, }\DecValTok{2}\NormalTok{)))}
\NormalTok{vision_model.add(Conv2D(}\DecValTok{128}\NormalTok{, (}\DecValTok{3}\NormalTok{, }\DecValTok{3}\NormalTok{), activation}\OperatorTok{=}\StringTok{'relu'}\NormalTok{, padding}\OperatorTok{=}\StringTok{'same'}\NormalTok{))}
\NormalTok{vision_model.add(Conv2D(}\DecValTok{128}\NormalTok{, (}\DecValTok{3}\NormalTok{, }\DecValTok{3}\NormalTok{), activation}\OperatorTok{=}\StringTok{'relu'}\NormalTok{))}
\NormalTok{vision_model.add(MaxPooling2D((}\DecValTok{2}\NormalTok{, }\DecValTok{2}\NormalTok{)))}
\NormalTok{vision_model.add(Conv2D(}\DecValTok{256}\NormalTok{, (}\DecValTok{3}\NormalTok{, }\DecValTok{3}\NormalTok{), activation}\OperatorTok{=}\StringTok{'relu'}\NormalTok{, padding}\OperatorTok{=}\StringTok{'same'}\NormalTok{))}
\NormalTok{vision_model.add(Conv2D(}\DecValTok{256}\NormalTok{, (}\DecValTok{3}\NormalTok{, }\DecValTok{3}\NormalTok{), activation}\OperatorTok{=}\StringTok{'relu'}\NormalTok{))}
\NormalTok{vision_model.add(Conv2D(}\DecValTok{256}\NormalTok{, (}\DecValTok{3}\NormalTok{, }\DecValTok{3}\NormalTok{), activation}\OperatorTok{=}\StringTok{'relu'}\NormalTok{))}
\NormalTok{vision_model.add(MaxPooling2D((}\DecValTok{2}\NormalTok{, }\DecValTok{2}\NormalTok{)))}
\NormalTok{vision_model.add(Flatten())}

\CommentTok{# 现在让我们用视觉模型来得到一个输出张量:}
\NormalTok{image_input }\OperatorTok{=} \NormalTok{Input(shape}\OperatorTok{=}\NormalTok{(}\DecValTok{224}\NormalTok{, }\DecValTok{224}\NormalTok{, }\DecValTok{3}\NormalTok{))}
\NormalTok{encoded_image }\OperatorTok{=} \NormalTok{vision_model(image_input)}

\CommentTok{# 接下来,定义一个语言模型来将问题编码成一个向量。}
\CommentTok{# 每个问题最长 100 个词,词的索引从 1 到 9999.}
\NormalTok{question_input }\OperatorTok{=} \NormalTok{Input(shape}\OperatorTok{=}\NormalTok{(}\DecValTok{100}\NormalTok{,), dtype}\OperatorTok{=}\StringTok{'int32'}\NormalTok{)}
\NormalTok{embedded_question }\OperatorTok{=} \NormalTok{Embedding(input_dim}\OperatorTok{=}\DecValTok{10000}\NormalTok{, output_dim}\OperatorTok{=}\DecValTok{256}, \\
\hspace{3cm}\NormalTok{input_length}\OperatorTok{=}\DecValTok{100}\NormalTok{)(question_input)}
\NormalTok{encoded_question }\OperatorTok{=} \NormalTok{LSTM(}\DecValTok{256}\NormalTok{)(embedded_question)}

\CommentTok{# 连接问题向量和图像向量:}
\NormalTok{merged }\OperatorTok{=} \NormalTok{keras.layers.concatenate([encoded_question, encoded_image])}

\CommentTok{# 然后在上面训练一个 1000 词的逻辑回归模型:}
\NormalTok{output }\OperatorTok{=} \NormalTok{Dense(}\DecValTok{1000}\NormalTok{, activation}\OperatorTok{=}\StringTok{'softmax'}\NormalTok{)(merged)}

\CommentTok{# 最终模型:}
\NormalTok{vqa_model }\OperatorTok{=} \NormalTok{Model(inputs}\OperatorTok{=}\NormalTok{[image_input, question_input], outputs}\OperatorTok{=}\NormalTok{output)}

\CommentTok{# 下一步就是在真实数据上训练模型。}
\end{Highlighting}
\end{Shaded}

\subsubsubsection{视频问答模型}\label{ux89c6ux9891ux95eeux7b54ux6a21ux578b}

现在我们已经训练了图像问答模型,我们可以很快地将它转换为视频问答模型。在适当的训练下,你可以给它展示一小段视频(例如
100
帧的人体动作),然后问它一个关于这段视频的问题(例如,「这个人在做什么运动?」
-\textgreater{} 「足球」)。

\begin{Shaded}
\begin{Highlighting}[]
\ImportTok{from} \NormalTok{keras.layers }\ImportTok{import} \NormalTok{TimeDistributed}

\NormalTok{video_input }\OperatorTok{=} \NormalTok{Input(shape}\OperatorTok{=}\NormalTok{(}\DecValTok{100}\NormalTok{, }\DecValTok{224}\NormalTok{, }\DecValTok{224}\NormalTok{, }\DecValTok{3}\NormalTok{))}
\CommentTok{# 这是基于之前定义的视觉模型(权重被重用)构建的视频编码}
\NormalTok{encoded_frame_sequence }\OperatorTok{=} \NormalTok{TimeDistributed(vision_model)(video_input)  }\CommentTok{# 输出为向量的序列}
\NormalTok{encoded_video }\OperatorTok{=} \NormalTok{LSTM(}\DecValTok{256}\NormalTok{)(encoded_frame_sequence)  }\CommentTok{# 输出为一个向量}

\CommentTok{# 这是问题编码器的模型级表示,重复使用与之前相同的权重:}
\NormalTok{question_encoder }\OperatorTok{=} \NormalTok{Model(inputs}\OperatorTok{=}\NormalTok{question_input, outputs}\OperatorTok{=}\NormalTok{encoded_question)}

\CommentTok{# 让我们用它来编码这个问题:}
\NormalTok{video_question_input }\OperatorTok{=} \NormalTok{Input(shape}\OperatorTok{=}\NormalTok{(}\DecValTok{100}\NormalTok{,), dtype}\OperatorTok{=}\StringTok{'int32'}\NormalTok{)}
\NormalTok{encoded_video_question }\OperatorTok{=} \NormalTok{question_encoder(video_question_input)}

\CommentTok{# 这就是我们的视频问答模式:}
\NormalTok{merged }\OperatorTok{=} \NormalTok{keras.layers.concatenate([encoded_video, encoded_video_question])}
\NormalTok{output }\OperatorTok{=} \NormalTok{Dense(}\DecValTok{1000}\NormalTok{, activation}\OperatorTok{=}\StringTok{'softmax'}\NormalTok{)(merged)}
\NormalTok{video_qa_model }\OperatorTok{=} \NormalTok{Model(inputs}\OperatorTok{=}\NormalTok{[video_input, video_question_input], outputs}\OperatorTok{=}\NormalTok{output)}
\end{Highlighting}
\end{Shaded}

\newpage
