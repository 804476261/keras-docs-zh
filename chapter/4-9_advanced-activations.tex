\subsection{高级激活层 Advanced Activations}\label{advanced-activations}

    
\subsubsection{LeakyReLU {\href{https://github.com/keras-team/keras/blob/master/keras/layers/advanced_activations.py\#L18}{{[}source{]}}}}

\begin{Shaded}
\begin{Highlighting}[]
\NormalTok{keras.layers.LeakyReLU(alpha}\OperatorTok{=}\FloatTok{0.3}\NormalTok{)}
\end{Highlighting}
\end{Shaded}

带泄漏的修正线性单元。

当神经元未激活时,它仍可以赋予其一个很小的梯度:
\texttt{f(x)\ =\ alpha\ *\ x\ for\ x\ \textless{}\ 0},
\texttt{f(x)\ =\ x\ for\ x\ \textgreater{}=\ 0}.

\textbf{输入尺寸}

可以是任意的。如果将该层作为模型的第一层, 则需要指定
\texttt{input\_shape} 参数 (整数元组,不包含样本数量的维度)。

\textbf{输出尺寸}

与输入相同。

\textbf{参数}

\begin{itemize}
\tightlist
\item
  \textbf{alpha}: float \textgreater{}= 0。负斜率系数。
\end{itemize}

\textbf{参考文献}

\begin{itemize}
\tightlist
\item
  \href{https://web.stanford.edu/~awni/papers/relu_hybrid_icml2013_final.pdf}{Rectifier
  Nonlinearities Improve Neural Network Acoustic Models}
\end{itemize}




\subsubsection{PReLU {\href{https://github.com/keras-team/keras/blob/master/keras/layers/advanced_activations.py\#L57}{{[}source{]}}}}

\begin{Shaded}
\begin{Highlighting}[]
\NormalTok{keras.layers.PReLU(alpha_initializer}\OperatorTok{=}\StringTok{'zeros'}\NormalTok{, alpha_regularizer}\OperatorTok{=}\VariableTok{None}, \\
\hspace{3cm}\NormalTok{alpha_constraint}\OperatorTok{=}\VariableTok{None}\NormalTok{, shared_axes}\OperatorTok{=}\VariableTok{None}\NormalTok{)}
\end{Highlighting}
\end{Shaded}

参数化的修正线性单元。

形式: \texttt{f(x)\ =\ alpha\ *\ x\ for\ x\ \textless{}\ 0},
\texttt{f(x)\ =\ x\ for\ x\ \textgreater{}=\ 0}, 其中 \texttt{alpha}
是一个可学习的数组,尺寸与 x 相同。

\textbf{输入尺寸}

可以是任意的。如果将这一层作为模型的第一层, 则需要指定
\texttt{input\_shape} 参数 (整数元组,不包含样本数量的维度)。

\textbf{输出尺寸}

与输入相同。

\textbf{参数}

\begin{itemize}
\tightlist
\item
  \textbf{alpha\_initializer}: 权重的初始化函数。
\item
  \textbf{alpha\_regularizer}: 权重的正则化方法。
\item
  \textbf{alpha\_constraint}: 权重的约束。
\item
  \textbf{shared\_axes}: 激活函数共享可学习参数的轴。
  例如,如果输入特征图来自输出形状为
  \texttt{(batch,\ height,\ width,\ channels)} 的 2D
  卷积层,而且你希望跨空间共享参数,以便每个滤波器只有一组参数, 可设置
  \texttt{shared\_axes={[}1,\ 2{]}}。
\end{itemize}

\textbf{参考文献}

\begin{itemize}
\tightlist
\item
  \href{https://arxiv.org/abs/1502.01852}{Delving Deep into Rectifiers:
  Surpassing Human-Level Performance on ImageNet Classification}
\end{itemize}




\subsubsection{ELU {\href{https://github.com/keras-team/keras/blob/master/keras/layers/advanced_activations.py\#L152}{{[}source{]}}}}

\begin{Shaded}
\begin{Highlighting}[]
\NormalTok{keras.layers.ELU(alpha}\OperatorTok{=}\FloatTok{1.0}\NormalTok{)}
\end{Highlighting}
\end{Shaded}

指数线性单元。

形式:
\texttt{f(x)\ =\ \ alpha\ *\ (exp(x)\ -\ 1.)\ for\ x\ \textless{}\ 0},
\texttt{f(x)\ =\ x\ for\ x\ \textgreater{}=\ 0}.

\textbf{输入尺寸}

可以是任意的。如果将这一层作为模型的第一层, 则需要指定
\texttt{input\_shape} 参数 (整数元组,不包含样本数量的维度)。

\textbf{输出尺寸}

与输入相同。

\textbf{参数}

\begin{itemize}
\tightlist
\item
  \textbf{alpha}: 负因子的尺度。
\end{itemize}

\textbf{参考文献}

\begin{itemize}
\tightlist
\item
  \href{https://arxiv.org/abs/1511.07289v1}{Fast and Accurate Deep
  Network Learning by Exponential Linear Units (ELUs)}
\end{itemize}



\subsubsection{ThresholdedReLU {\href{https://github.com/keras-team/keras/blob/master/keras/layers/advanced_activations.py\#L191}{{[}source{]}}}}

\begin{Shaded}
\begin{Highlighting}[]
\NormalTok{keras.layers.ThresholdedReLU(theta}\OperatorTok{=}\FloatTok{1.0}\NormalTok{)}
\end{Highlighting}
\end{Shaded}

带阈值的修正线性单元。

形式: \texttt{f(x)\ =\ x\ for\ x\ \textgreater{}\ theta},
\texttt{f(x)\ =\ 0\ otherwise}.

\textbf{输入尺寸}

可以是任意的。如果将这一层作为模型的第一层, 则需要指定
\texttt{input\_shape} 参数 (整数元组,不包含样本数量的维度)。

\textbf{输出尺寸}

与输入相同。

\textbf{参数}

\begin{itemize}
\tightlist
\item
  \textbf{theta}: float \textgreater{}= 0。激活的阈值位。
\end{itemize}

\textbf{参考文献}

\begin{itemize}
\tightlist
\item
  \href{http://arxiv.org/abs/1402.3337}{Zero-Bias Autoencoders and the
  Benefits of Co-Adapting Features}
\end{itemize}



\subsubsection{Softmax {\href{https://github.com/keras-team/keras/blob/master/keras/layers/advanced_activations.py\#L230}{{[}source{]}}}}

\begin{Shaded}
\begin{Highlighting}[]
\NormalTok{keras.layers.Softmax(axis}\OperatorTok{=-}\DecValTok{1}\NormalTok{)}
\end{Highlighting}
\end{Shaded}

Softmax 激活函数。

\textbf{输入尺寸}

可以是任意的。如果将这一层作为模型的第一层, 则需要指定
\texttt{input\_shape} 参数 (整数元组,不包含样本数量的维度)。

\textbf{输出尺寸}

与输入相同。

\textbf{参数}

\begin{itemize}
\tightlist
\item
  \textbf{axis}: 整数,应用 softmax 标准化的轴。
\end{itemize}

\newpage
