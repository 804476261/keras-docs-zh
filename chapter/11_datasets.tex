
\section{常用数据集 Datasets}\label{datasets}

\subsection{CIFAR10
小图像分类数据集}\label{cifar10-ux5c0fux56feux50cfux5206ux7c7bux6570ux636eux96c6}

50,000 张 32x32 彩色训练图像数据,以及 10,000 张测试图像数据,总共分为
10 个类别。

\textbf{用法:}\label{ux7528ux6cd5}

\begin{Shaded}
\begin{Highlighting}[]
\ImportTok{from} \NormalTok{keras.datasets }\ImportTok{import} \NormalTok{cifar10}

\NormalTok{(x_train, y_train), (x_test, y_test) }\OperatorTok{=} \NormalTok{cifar10.load_data()}
\end{Highlighting}
\end{Shaded}

\begin{itemize}
\tightlist
\item
  \textbf{返回:}
\item
  2 个元组:

  \begin{itemize}
  \tightlist
  \item
    \textbf{x\_train, x\_test}: uint8 数组表示的 RGB 图像数据,尺寸为
    (num\_samples, 3, 32, 32)。
  \item
    \textbf{y\_train, y\_test}: uint8 数组表示的类别标签(范围在 0-9
    之间的整数),尺寸为 (num\_samples,)。
  \end{itemize}
\end{itemize}


\subsection{CIFAR100
小图像分类数据集}\label{cifar100-ux5c0fux56feux50cfux5206ux7c7bux6570ux636eux96c6}

50,000 张 32x32 彩色训练图像数据,以及 10,000 张测试图像数据,总共分为
100 个类别。

\textbf{用法:}\label{ux7528ux6cd5-1}

\begin{Shaded}
\begin{Highlighting}[]
\ImportTok{from} \NormalTok{keras.datasets }\ImportTok{import} \NormalTok{cifar100}

\NormalTok{(x_train, y_train), (x_test, y_test) }\OperatorTok{=} \NormalTok{cifar100.load_data(label_mode}\OperatorTok{=}\StringTok{'fine'}\NormalTok{)}
\end{Highlighting}
\end{Shaded}

\begin{itemize}
\tightlist
\item
  \textbf{返回:}
\item
  2 个元组:

  \begin{itemize}
  \tightlist
  \item
    \textbf{x\_train, x\_test}: uint8 数组表示的 RGB 图像数据,尺寸为
    (num\_samples, 3, 32, 32)。
  \item
    \textbf{y\_train, y\_test}: uint8 数组表示的类别标签(范围在 0-9
    之间的整数),尺寸为 (num\_samples,)。
  \end{itemize}
\item
  \textbf{参数:}
\item
  \textbf{label\_mode}: "fine" 或者 "coarse"
\end{itemize}


\subsection{IMDB
电影评论情感分类数据集}\label{imdb-ux7535ux5f71ux8bc4ux8bbaux60c5ux611fux5206ux7c7bux6570ux636eux96c6}

数据集来自 IMDB 的 25,000
条电影评论,以情绪(正面/负面)标记。每一条评论已经过预处理,并编码为词索引(整数)的\hyperref[preprocessing-sequence]{序列}表示。为了方便起见,将词按数据集中出现的频率进行索引,例如整数
3 编码数据中第三个最频繁的词。这允许快速筛选操作,例如:「只考虑前
10,000 个最常用的词,但排除前 20 个最常见的词」。

作为惯例,0 不代表特定的单词,而是被用于编码任何未知单词。

\textbf{用法}\label{ux7528ux6cd5-2}

\begin{Shaded}
\begin{Highlighting}[]
\ImportTok{from} \NormalTok{keras.datasets }\ImportTok{import} \NormalTok{imdb}

\NormalTok{(x_train, y_train), (x_test, y_test) }\OperatorTok{=} \NormalTok{imdb.load_data(path}\OperatorTok{=}\StringTok{"imdb.npz"}\NormalTok{,}
                                                      \NormalTok{num_words}\OperatorTok{=}\VariableTok{None}\NormalTok{,}
                                                      \NormalTok{skip_top}\OperatorTok{=}\DecValTok{0}\NormalTok{,}
                                                      \NormalTok{maxlen}\OperatorTok{=}\VariableTok{None}\NormalTok{,}
                                                      \NormalTok{seed}\OperatorTok{=}\DecValTok{113}\NormalTok{,}
                                                      \NormalTok{start_char}\OperatorTok{=}\DecValTok{1}\NormalTok{,}
                                                      \NormalTok{oov_char}\OperatorTok{=}\DecValTok{2}\NormalTok{,}
                                                      \NormalTok{index_from}\OperatorTok{=}\DecValTok{3}\NormalTok{)}
\end{Highlighting}
\end{Shaded}

\begin{itemize}
\tightlist
\item
  \textbf{返回:}
\item
  2 个元组:

  \begin{itemize}
  \tightlist
  \item
    \textbf{x\_train, x\_test}: 序列的列表,即词索引的列表。如果指定了
    \texttt{num\_words} 参数,则可能的最大索引值是
    \texttt{num\_words-1}。如果指定了 \texttt{maxlen}
    参数,则可能的最大序列长度为 \texttt{maxlen}。
  \item
    \textbf{y\_train, y\_test}: 整数标签列表 (1 或 0)。
  \end{itemize}
\item
  \textbf{参数:}
\item
  \textbf{path}: 如果你本地没有该数据集 (在
  \texttt{\textquotesingle{}\textasciitilde{}/.keras/datasets/\textquotesingle{}\ +\ path}),它将被下载到此目录。
\item
  \textbf{num\_words}: 整数或
  None。要考虑的最常用的词语。任何不太频繁的词将在序列数据中显示为
  \texttt{oov\_char} 值。
\item
  \textbf{skip\_top}:
  整数。要忽略的最常见的单词(它们将在序列数据中显示为
  \texttt{oov\_char} 值)。
\item
  \textbf{maxlen}: 整数。最大序列长度。 任何更长的序列都将被截断。
\item
  \textbf{seed}: 整数。用于可重现数据混洗的种子。
\item
  \textbf{start\_char}: 整数。序列的开始将用这个字符标记。设置为 1,因为
  0 通常作为填充字符。
\item
  \textbf{oov\_char}: 整数。由于 \texttt{num\_words} 或
  \texttt{skip\_top} 限制而被删除的单词将被替换为此字符。
\item
  \textbf{index\_from}:
  整数。使用此数以上更高的索引值实际词汇索引的开始。
\end{itemize}


\subsection{路透社新闻主题分类}\label{ux8defux900fux793eux65b0ux95fbux4e3bux9898ux5206ux7c7b}

数据集来源于路透社的 11,228 条新闻文本,总共分为 46 个主题。与 IMDB
数据集一样,每条新闻都被编码为一个词索引的序列(相同的约定)。

\textbf{用法:}\label{ux7528ux6cd5-3}

\begin{Shaded}
\begin{Highlighting}[]
\ImportTok{from} \NormalTok{keras.datasets }\ImportTok{import} \NormalTok{reuters}

\NormalTok{(x_train, y_train), (x_test, y_test) }\OperatorTok{=} \NormalTok{reuters.load_data(path}\OperatorTok{=}\StringTok{"reuters.npz"}\NormalTok{,}
                                                         \NormalTok{num_words}\OperatorTok{=}\VariableTok{None}\NormalTok{,}
                                                         \NormalTok{skip_top}\OperatorTok{=}\DecValTok{0}\NormalTok{,}
                                                         \NormalTok{maxlen}\OperatorTok{=}\VariableTok{None}\NormalTok{,}
                                                         \NormalTok{test_split}\OperatorTok{=}\FloatTok{0.2}\NormalTok{,}
                                                         \NormalTok{seed}\OperatorTok{=}\DecValTok{113}\NormalTok{,}
                                                         \NormalTok{start_char}\OperatorTok{=}\DecValTok{1}\NormalTok{,}
                                                         \NormalTok{oov_char}\OperatorTok{=}\DecValTok{2}\NormalTok{,}
                                                         \NormalTok{index_from}\OperatorTok{=}\DecValTok{3}\NormalTok{)}
\end{Highlighting}
\end{Shaded}

规格与 IMDB 数据集的规格相同,但增加了:

\begin{itemize}
\tightlist
\item
  \textbf{test\_split}: 浮点型。用作测试集的数据比例。
\end{itemize}

该数据集还提供了用于编码序列的词索引:

\begin{Shaded}
\begin{Highlighting}[]
\NormalTok{word_index }\OperatorTok{=} \NormalTok{reuters.get_word_index(path}\OperatorTok{=}\StringTok{"reuters_word_index.json"}\NormalTok{)}
\end{Highlighting}
\end{Shaded}

\begin{itemize}
\item
  \textbf{返回:} 一个字典,其中键是单词(字符串),值是索引(整数)。
  例如,\texttt{word\_index{[}"giraffe"{]}} 可能会返回 \texttt{1234}。
\item
  \textbf{参数:}
\item
  \textbf{path}: 如果在本地没有索引文件 (at
  \texttt{\textquotesingle{}\textasciitilde{}/.keras/datasets/\textquotesingle{}\ +\ path}),
  它将被下载到该目录。
\end{itemize}


\subsection{MNIST
手写字符数据集}\label{mnist-ux624bux5199ux5b57ux7b26ux6570ux636eux96c6}

训练集为 60,000 张 28x28 像素灰度图像,测试集为 10,000 同规格图像,总共
10 类数字标签。

\textbf{用法:}\label{ux7528ux6cd5-4}

\begin{Shaded}
\begin{Highlighting}[]
\ImportTok{from} \NormalTok{keras.datasets }\ImportTok{import} \NormalTok{mnist}

\NormalTok{(x_train, y_train), (x_test, y_test) }\OperatorTok{=} \NormalTok{mnist.load_data()}
\end{Highlighting}
\end{Shaded}

\begin{itemize}
\tightlist
\item
  \textbf{返回:}
\item
  2 个元组:

  \begin{itemize}
  \tightlist
  \item
    \textbf{x\_train, x\_test}: uint8 数组表示的灰度图像,尺寸为
    (num\_samples, 28, 28)。
  \item
    \textbf{y\_train, y\_test}: uint8 数组表示的数字标签(范围在 0-9
    之间的整数),尺寸为 (num\_samples,)。
  \end{itemize}
\item
  \textbf{参数:}
\item
  \textbf{path}: 如果在本地没有索引文件 (at
  \texttt{\textquotesingle{}\textasciitilde{}/.keras/datasets/\textquotesingle{}\ +\ path}),
  它将被下载到该目录。
\end{itemize}


\subsection{Fashion-MNIST
时尚物品数据集}\label{fashion-mnist-ux65f6ux5c1aux7269ux54c1ux6570ux636eux96c6}

训练集为 60,000 张 28x28 像素灰度图像,测试集为 10,000 同规格图像,总共
10 类时尚物品标签。该数据集可以用作 MNIST 的直接替代品。类别标签是:

\begin{longtable}[c]{@{}lll@{}}
\toprule
类别 & 描述 & 中文\tabularnewline
\midrule
\endhead
0 & T-shirt/top & T恤/上衣\tabularnewline
1 & Trouser & 裤子\tabularnewline
2 & Pullover & 套头衫\tabularnewline
3 & Dress & 连衣裙\tabularnewline
4 & Coat & 外套\tabularnewline
5 & Sandal & 凉鞋\tabularnewline
6 & Shirt & 衬衫\tabularnewline
7 & Sneaker & 运动鞋\tabularnewline
8 & Bag & 背包\tabularnewline
9 & Ankle boot & 短靴\tabularnewline
\bottomrule
\end{longtable}

\textbf{用法:}\label{ux7528ux6cd5-5}

\begin{Shaded}
\begin{Highlighting}[]
\ImportTok{from} \NormalTok{keras.datasets }\ImportTok{import} \NormalTok{fashion_mnist}

\NormalTok{(x_train, y_train), (x_test, y_test) }\OperatorTok{=} \NormalTok{fashion_mnist.load_data()}
\end{Highlighting}
\end{Shaded}

\begin{itemize}
\tightlist
\item
  \textbf{返回:}
\item
  2 个元组:

  \begin{itemize}
  \tightlist
  \item
    \textbf{x\_train, x\_test}: uint8 数组表示的灰度图像,尺寸为
    (num\_samples, 28, 28)。
  \item
    \textbf{y\_train, y\_test}: uint8 数组表示的数字标签(范围在 0-9
    之间的整数),尺寸为 (num\_samples,)。
  \end{itemize}
\end{itemize}


\subsection{Boston
房价回归数据集}\label{boston-ux623fux4ef7ux56deux5f52ux6570ux636eux96c6}

数据集来自卡内基梅隆大学维护的 StatLib 库。

样本包含 1970 年代的在波士顿郊区不同位置的房屋信息,总共有 13
种房屋属性。 目标值是一个位置的房屋的中值(单位:k\$)。

\textbf{用法:}\label{ux7528ux6cd5-6}

\begin{Shaded}
\begin{Highlighting}[]
\ImportTok{from} \NormalTok{keras.datasets }\ImportTok{import} \NormalTok{boston_housing}

\NormalTok{(x_train, y_train), (x_test, y_test) }\OperatorTok{=} \NormalTok{boston_housing.load_data()}
\end{Highlighting}
\end{Shaded}

\begin{itemize}
\item
  \textbf{参数:}
\item
  \textbf{path}: 缓存本地数据集的位置 (相对路径
  \textasciitilde{}/.keras/datasets)。
\item
  \textbf{seed}: 在计算测试分割之前对数据进行混洗的随机种子。
\item
  \textbf{test\_split}: 需要保留作为测试数据的比例。
\item
  \textbf{返回:} Numpy 数组的元组:
  \texttt{(x\_train,\ y\_train),\ (x\_test,\ y\_test)}。
\end{itemize}

\newpage
