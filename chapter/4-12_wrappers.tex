
\subsection{层封装器 wrappers  }
\subsubsection{TimeDistributed {\href{https://github.com/keras-team/keras/blob/master/keras/layers/wrappers.py\#L104}{{[}source{]}}}}

\begin{Shaded}
\begin{Highlighting}[]
\NormalTok{keras.layers.TimeDistributed(layer)}
\end{Highlighting}
\end{Shaded}

这个封装器将一个层应用于输入的每个时间片。

输入至少为 3D,且第一个维度应该是时间所表示的维度。

考虑 32 个样本的一个 batch, 其中每个样本是 10 个 16 维向量的序列。
那么这个 batch 的输入尺寸为 \texttt{(32,\ 10,\ 16)}, 而
\texttt{input\_shape} 不包含样本数量的维度,为 \texttt{(10,\ 16)}。

你可以使用 \texttt{TimeDistributed} 来将 \texttt{Dense} 层独立地应用到
这 10 个时间步的每一个:

\begin{Shaded}
\begin{Highlighting}[]
\CommentTok{# 作为模型第一层}
\NormalTok{model }\OperatorTok{=} \NormalTok{Sequential()}
\NormalTok{model.add(TimeDistributed(Dense(}\DecValTok{8}\NormalTok{), input_shape}\OperatorTok{=}\NormalTok{(}\DecValTok{10}\NormalTok{, }\DecValTok{16}\NormalTok{)))}
\CommentTok{# 现在 model.output_shape == (None, 10, 8)}
\end{Highlighting}
\end{Shaded}

输出的尺寸为 \texttt{(32,\ 10,\ 8)}。

在后续的层中,将不再需要 \texttt{input\_shape}:

\begin{Shaded}
\begin{Highlighting}[]
\NormalTok{model.add(TimeDistributed(Dense(}\DecValTok{32}\NormalTok{)))}
\CommentTok{# 现在 model.output_shape == (None, 10, 32)}
\end{Highlighting}
\end{Shaded}

输出的尺寸为 \texttt{(32,\ 10,\ 32)}。

\texttt{TimeDistributed} 可以应用于任意层,不仅仅是 \texttt{Dense},
例如运用于 \texttt{Conv2D} 层:

\begin{Shaded}
\begin{Highlighting}[]
\NormalTok{model }\OperatorTok{=} \NormalTok{Sequential()}
\NormalTok{model.add(TimeDistributed(Conv2D(}\DecValTok{64}\NormalTok{, (}\DecValTok{3}\NormalTok{, }\DecValTok{3}\NormalTok{)),}
                          \NormalTok{input_shape}\OperatorTok{=}\NormalTok{(}\DecValTok{10}\NormalTok{, }\DecValTok{299}\NormalTok{, }\DecValTok{299}\NormalTok{, }\DecValTok{3}\NormalTok{)))}
\end{Highlighting}
\end{Shaded}

\textbf{参数}

\begin{itemize}
\tightlist
\item
  \textbf{layer}: 一个网络层实例。
\end{itemize}



\subsubsection{Bidirectional {\href{https://github.com/keras-team/keras/blob/master/keras/layers/wrappers.py\#L221}{{[}source{]}}}}

\begin{Shaded}
\begin{Highlighting}[]
\NormalTok{keras.layers.Bidirectional(layer, merge_mode}\OperatorTok{=}\StringTok{'concat'}\NormalTok{, weights}\OperatorTok{=}\VariableTok{None}\NormalTok{)}
\end{Highlighting}
\end{Shaded}

RNN 的双向封装器,对序列进行前向和后向计算。

\textbf{参数}

\begin{itemize}
\tightlist
\item
  \textbf{layer}: \texttt{Recurrent} 实例。
\item
  \textbf{merge\_mode}: 前向和后向 RNN 的输出的结合模式。 为 \{'sum',
  'mul', 'concat', 'ave', None\} 其中之一。 如果是
  None,输出不会被结合,而是作为一个列表被返回。
\end{itemize}

\textbf{异常}

\begin{itemize}
\tightlist
\item
  \textbf{ValueError}: 如果参数 \texttt{merge\_mode} 非法。
\end{itemize}

\textbf{例}

\begin{Shaded}
\begin{Highlighting}[]
\NormalTok{model }\OperatorTok{=} \NormalTok{Sequential()}
\NormalTok{model.add(Bidirectional(LSTM(}\DecValTok{10}\NormalTok{, return_sequences}\OperatorTok{=}\VariableTok{True}\NormalTok{),}
                        \NormalTok{input_shape}\OperatorTok{=}\NormalTok{(}\DecValTok{5}\NormalTok{, }\DecValTok{10}\NormalTok{)))}
\NormalTok{model.add(Bidirectional(LSTM(}\DecValTok{10}\NormalTok{)))}
\NormalTok{model.add(Dense(}\DecValTok{5}\NormalTok{))}
\NormalTok{model.add(Activation(}\StringTok{'softmax'}\NormalTok{))}
\NormalTok{model.}\BuiltInTok{compile}\NormalTok{(loss}\OperatorTok{=}\StringTok{'categorical_crossentropy'}\NormalTok{, optimizer}\OperatorTok{=}\StringTok{'rmsprop'}\NormalTok{)}
\end{Highlighting}
\end{Shaded}

\newpage