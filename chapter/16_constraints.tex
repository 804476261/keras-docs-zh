\section{约束 Constraints}\label{constraints}
\subsection{约束使用}
来自\texttt{constraints}模块的函数允许在优化期间对网络参数设置约束(例如,非负)。

惩罚以每层为基础进行。
确切的API将取决于层,但层\texttt{密集},\texttt{Conv1D},\texttt{Conv2D}和\texttt{Conv3D}具有统一的API。
这些图层显示2个关键字参数:

\begin{itemize}
\tightlist
\item
  \texttt{kernel\_constraint} 为主权重矩阵。
\item
  \texttt{bias\_constraint} 为偏见。
\end{itemize}

\begin{Shaded}
\begin{Highlighting}[]
\ImportTok{from} \NormalTok{keras.constraints }\ImportTok{import} \NormalTok{max_norm}
\NormalTok{model.add(Dense(}\DecValTok{64}\NormalTok{, kernel_constraint}\OperatorTok{=}\NormalTok{max_norm(}\DecValTok{2}\NormalTok{.)))}
\end{Highlighting}
\end{Shaded}

\subsection{可用的约束}\label{ux53efux7528ux7684ux7ea6ux675f}

\begin{itemize}
\tightlist
\item
  \textbf{max\_norm(max\_value=2, axis=0)}: 最大范数约束
\item
  \textbf{non\_neg()}: 非负面约束
\item
  \textbf{unit\_norm(axis=0)}: 单位规范约束
\item
  \textbf{min\_max\_norm(min\_value=0.0, max\_value=1.0, rate=1.0,
  axis=0)}: 最小/最大范数约束
\end{itemize}
\newpage