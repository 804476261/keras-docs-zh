\section{优化器 Optimizers}\label{optimizers}
\subsection{优化器的用法}

优化器(optimizer)是编译Keras模型的所需的两个参数之一:

\begin{Shaded}
\begin{Highlighting}[]
\ImportTok{from} \NormalTok{keras }\ImportTok{import} \NormalTok{optimizers}

\NormalTok{model }\OperatorTok{=} \NormalTok{Sequential()}
\NormalTok{model.add(Dense(}\DecValTok{64}\NormalTok{, kernel_initializer}\OperatorTok{=}\StringTok{'uniform'}\NormalTok{, input_shape}\OperatorTok{=}\NormalTok{(}\DecValTok{10}\NormalTok{,)))}
\NormalTok{model.add(Activation(}\StringTok{'tanh'}\NormalTok{))}
\NormalTok{model.add(Activation(}\StringTok{'softmax'}\NormalTok{))}

\NormalTok{sgd }\OperatorTok{=} \NormalTok{optimizers.SGD(lr}\OperatorTok{=}\FloatTok{0.01}\NormalTok{, decay}\OperatorTok{=}\FloatTok{1e-6}\NormalTok{, momentum}\OperatorTok{=}\FloatTok{0.9}\NormalTok{, nesterov}\OperatorTok{=}\VariableTok{True}\NormalTok{)}
\NormalTok{model.}\BuiltInTok{compile}\NormalTok{(loss}\OperatorTok{=}\StringTok{'mean_squared_error'}\NormalTok{, optimizer}\OperatorTok{=}\NormalTok{sgd)}
\end{Highlighting}
\end{Shaded}

你可以先实例化一个优化器对象,然后将它传入\texttt{model.compile()},像上述示例中一样,
或者你可以通过名称来调用优化器。在后一种情况下,将使用优化器的默认参数。

\begin{Shaded}
\begin{Highlighting}[]
\CommentTok{# 传入优化器名称: 默认参数将被采用}
\NormalTok{model.}\BuiltInTok{compile}\NormalTok{(loss}\OperatorTok{=}\StringTok{'mean_squared_error'}\NormalTok{, optimizer}\OperatorTok{=}\StringTok{'sgd'}\NormalTok{)}
\end{Highlighting}
\end{Shaded}


\subsection{Keras优化器的公共参数}\label{kerasux4f18ux5316ux5668ux7684ux516cux5171ux53c2ux6570}

参数\texttt{clipnorm}和\texttt{clipvalue}能在所有的优化器中使用,用于控制梯度裁剪(Gradient
Clipping):

\begin{Shaded}
\begin{Highlighting}[]
\ImportTok{from} \NormalTok{keras }\ImportTok{import} \NormalTok{optimizers}

\CommentTok{# 所有参数梯度将被裁剪,让其l2范数最大为1:g * 1 / max(1, l2_norm)}
\NormalTok{sgd }\OperatorTok{=} \NormalTok{optimizers.SGD(lr}\OperatorTok{=}\FloatTok{0.01}\NormalTok{, clipnorm}\OperatorTok{=}\DecValTok{1}\NormalTok{.)}
\end{Highlighting}
\end{Shaded}

\begin{Shaded}
\begin{Highlighting}[]
\ImportTok{from} \NormalTok{keras }\ImportTok{import} \NormalTok{optimizers}

\CommentTok{# 所有参数d 梯度将被裁剪到数值范围内:}
\CommentTok{# 最大值0.5}
\CommentTok{# 最小值-0.5}
\NormalTok{sgd }\OperatorTok{=} \NormalTok{optimizers.SGD(lr}\OperatorTok{=}\FloatTok{0.01}\NormalTok{, clipvalue}\OperatorTok{=}\FloatTok{0.5}\NormalTok{)}
\end{Highlighting}
\end{Shaded}


 
\subsubsection{SGD {\href{https://github.com/keras-team/keras/blob/master/keras/optimizers.py\#L135}{{[}source{]}}}}

\begin{Shaded}
\begin{Highlighting}[]
\NormalTok{keras.optimizers.SGD(lr}\OperatorTok{=}\FloatTok{0.01}\NormalTok{, momentum}\OperatorTok{=}\FloatTok{0.0}\NormalTok{, decay}\OperatorTok{=}\FloatTok{0.0}\NormalTok{, nesterov}\OperatorTok{=}\VariableTok{False}\NormalTok{)}
\end{Highlighting}
\end{Shaded}

随机梯度下降优化器

包含扩展功能的支持: - 动量(momentum)优化, -
学习率衰减(每次参数更新后) - Nestrov动量(NAG)优化

\textbf{参数}

\begin{itemize}
\tightlist
\item
  \textbf{lr}: float \textgreater{}= 0. 学习率
\item
  \textbf{momentum}: float \textgreater{}= 0.
  参数,用于加速SGD在相关方向上前进,并抑制震荡
\item
  \textbf{decay}: float \textgreater{}= 0. 每次参数更新后学习率衰减值.
\item
  \textbf{nesterov}: boolean. 是否使用Nesterov动量.
\end{itemize}



\subsubsection{RMSprop {\href{https://github.com/keras-team/keras/blob/master/keras/optimizers.py\#L198}{{[}source{]}}}}

\begin{Shaded}
\begin{Highlighting}[]
\NormalTok{keras.optimizers.RMSprop(lr}\OperatorTok{=}\FloatTok{0.001}\NormalTok{, rho}\OperatorTok{=}\FloatTok{0.9}\NormalTok{, epsilon}\OperatorTok{=}\VariableTok{None}\NormalTok{, decay}\OperatorTok{=}\FloatTok{0.0}\NormalTok{)}
\end{Highlighting}
\end{Shaded}

RMSProp优化器.

建议使用优化器的默认参数 (除了学习率lr,它可以被自由调节)

这个优化器通常是训练循环神经网络RNN的不错选择。

\textbf{参数}

\begin{itemize}
\tightlist
\item
  \textbf{lr}: float \textgreater{}= 0. 学习率.
\item
  \textbf{rho}: float \textgreater{}= 0.
  RMSProp梯度平方的移动均值的衰减率.
\item
  \textbf{epsilon}: float \textgreater{}= 0. 模糊因子. 若为
  \texttt{None}, 默认为 \texttt{K.epsilon()}.
\item
  \textbf{decay}: float \textgreater{}= 0. 每次参数更新后学习率衰减值.
\end{itemize}

\textbf{引用}

\begin{itemize}
\tightlist
\item
  \href{http://www.cs.toronto.edu/~tijmen/csc321/slides/lecture_slides_lec6.pdf}{rmsprop:
  Divide the gradient by a running average of its recent magnitude}
\end{itemize}




\subsubsection{Adagrad {\href{https://github.com/keras-team/keras/blob/master/keras/optimizers.py\#L265}{{[}source{]}}}}

\begin{Shaded}
\begin{Highlighting}[]
\NormalTok{keras.optimizers.Adagrad(lr}\OperatorTok{=}\FloatTok{0.01}\NormalTok{, epsilon}\OperatorTok{=}\VariableTok{None}\NormalTok{, decay}\OperatorTok{=}\FloatTok{0.0}\NormalTok{)}
\end{Highlighting}
\end{Shaded}

Adagrad优化器.

建议使用优化器的默认参数。

\textbf{参数}

\begin{itemize}
\tightlist
\item
  \textbf{lr}: float \textgreater{}= 0. 学习率.
\item
  \textbf{epsilon}: float \textgreater{}= 0. 若为 \texttt{None}, 默认为
  \texttt{K.epsilon()}.
\item
  \textbf{decay}: float \textgreater{}= 0. 每次参数更新后学习率衰减值.
\end{itemize}

\textbf{引用}

\begin{itemize}
\tightlist
\item
  \href{http://www.jmlr.org/papers/volume12/duchi11a/duchi11a.pdf}{Adaptive
  Subgradient Methods for Online Learning and Stochastic Optimization}
\end{itemize}




\subsubsection{Adadelta {\href{https://github.com/keras-team/keras/blob/master/keras/optimizers.py\#L324}{{[}source{]}}}}

\begin{Shaded}
\begin{Highlighting}[]
\NormalTok{keras.optimizers.Adadelta(lr}\OperatorTok{=}\FloatTok{1.0}\NormalTok{, rho}\OperatorTok{=}\FloatTok{0.95}\NormalTok{, epsilon}\OperatorTok{=}\VariableTok{None}\NormalTok{, decay}\OperatorTok{=}\FloatTok{0.0}\NormalTok{)}
\end{Highlighting}
\end{Shaded}

Adagrad优化器.

建议使用优化器的默认参数。

\textbf{参数}

\begin{itemize}
\tightlist
\item
  \textbf{lr}: float \textgreater{}= 0. 学习率,建议保留默认值.
\item
  \textbf{rho}: float \textgreater{}= 0.
  Adadelta梯度平方移动均值的衰减率
\item
  \textbf{epsilon}: float \textgreater{}= 0. 模糊因子. 若为
  \texttt{None}, 默认为 \texttt{K.epsilon()}.
\item
  \textbf{decay}: float \textgreater{}= 0. 每次参数更新后学习率衰减值.
\end{itemize}

\textbf{引用}

\begin{itemize}
\tightlist
\item
  \href{http://arxiv.org/abs/1212.5701}{Adadelta - an adaptive learning
  rate method}
\end{itemize}




\subsubsection{Adam {\href{https://github.com/keras-team/keras/blob/master/keras/optimizers.py\#L397}{{[}source{]}}}}

\begin{Shaded}
\begin{Highlighting}[]
\NormalTok{keras.optimizers.Adam(lr}\OperatorTok{=}\FloatTok{0.001}\NormalTok{, beta_1}\OperatorTok{=}\FloatTok{0.9}\NormalTok{, beta_2}\OperatorTok{=}\FloatTok{0.999}\NormalTok{, epsilon}\OperatorTok{=}\VariableTok{None}, \\
\hspace{4cm}\NormalTok{decay}\OperatorTok{=}\FloatTok{0.0}\NormalTok{, amsgrad}\OperatorTok{=}\VariableTok{False}\NormalTok{)}
\end{Highlighting}
\end{Shaded}

Adam优化器.

默认参数遵循原论文中提供的值。

\textbf{参数}

\begin{itemize}
\tightlist
\item
  \textbf{lr}: float \textgreater{}= 0. 学习率.
\item
  \textbf{beta\_1}: float, 0 \textless{} beta \textless{} 1. 通常接近于
  1.
\item
  \textbf{beta\_2}: float, 0 \textless{} beta \textless{} 1. 通常接近于
  1.
\item
  \textbf{epsilon}: float \textgreater{}= 0. 模糊因子. 若为
  \texttt{None}, 默认为 \texttt{K.epsilon()}.
\item
  \textbf{decay}: float \textgreater{}= 0. 每次参数更新后学习率衰减值.
\item
  \textbf{amsgrad}: boolean. 是否应用此算法的AMSGrad变种,来自论文"On
  the Convergence of Adam and Beyond".
\end{itemize}

\textbf{引用}

\begin{itemize}
\tightlist
\item
  \href{http://arxiv.org/abs/1412.6980v8}{Adam - A Method for Stochastic
  Optimization}
\item
  \href{https://openreview.net/forum?id=ryQu7f-RZ}{On the Convergence of
  Adam and Beyond}
\end{itemize}




\subsubsection{Adamax {\href{https://github.com/keras-team/keras/blob/master/keras/optimizers.py\#L486}{{[}source{]}}}}

\begin{Shaded}
\begin{Highlighting}[]
\NormalTok{keras.optimizers.Adamax(lr}\OperatorTok{=}\FloatTok{0.002}\NormalTok{, beta_1}\OperatorTok{=}\FloatTok{0.9}\NormalTok{, beta_2}\OperatorTok{=}\FloatTok{0.999}\NormalTok{, epsilon}\OperatorTok{=}\VariableTok{None}\NormalTok{, decay}\OperatorTok{=}\FloatTok{0.0}\NormalTok{)}
\end{Highlighting}
\end{Shaded}

Adamax优化器,来自Adam论文的第七小节.

它是Adam算法基于无穷范数(infinity norm)的变种。
默认参数遵循论文中提供的值。

\textbf{参数}

\begin{itemize}
\tightlist
\item
  \textbf{lr}: float \textgreater{}= 0. 学习率.
\item
  \textbf{beta\_1/beta\_2}: floats, 0 \textless{} beta \textless{} 1.
  通常接近于 1.
\item
  \textbf{epsilon}: float \textgreater{}= 0. 模糊因子. 若为
  \texttt{None}, 默认为 \texttt{K.epsilon()}.
\item
  \textbf{decay}: float \textgreater{}= 0. 每次参数更新后学习率衰减值.
\end{itemize}

\textbf{引用}

\begin{itemize}
\tightlist
\item
  \href{http://arxiv.org/abs/1412.6980v8}{Adam - A Method for Stochastic
  Optimization}
\end{itemize}




\subsubsection{Nadam {\href{https://github.com/keras-team/keras/blob/master/keras/optimizers.py\#L563}{{[}source{]}}}}

\begin{Shaded}
\begin{Highlighting}[]
\NormalTok{keras.optimizers.Nadam(lr}\OperatorTok{=}\FloatTok{0.002}\NormalTok{, beta_1}\OperatorTok{=}\FloatTok{0.9}\NormalTok{, beta_2}\OperatorTok{=}\FloatTok{0.999}\NormalTok{, epsilon}\OperatorTok{=}\VariableTok{None}, \\
\hspace{4cm}\NormalTok{schedule_decay}\OperatorTok{=}\FloatTok{0.004}\NormalTok{)}
\end{Highlighting}
\end{Shaded}

Nesterov版本Adam优化器.

正像Adam本质上是RMSProp与动量momentum的结合, Nadam是采用Nesterov
momentum版本的Adam优化器。

默认参数遵循论文中提供的值。 建议使用优化器的默认参数。

\textbf{参数}

\begin{itemize}
\tightlist
\item
  \textbf{lr}: float \textgreater{}= 0. 学习率.
\item
  \textbf{beta\_1/beta\_2}: floats, 0 \textless{} beta \textless{} 1.
  通常接近于 1.
\item
  \textbf{epsilon}: float \textgreater{}= 0. 模糊因子. 若为
  \texttt{None}, 默认为 \texttt{K.epsilon()}.
\end{itemize}

\textbf{引用}

\begin{itemize}
\tightlist
\item
  \href{http://cs229.stanford.edu/proj2015/054_report.pdf}{Nadam report}
\item
  \href{http://www.cs.toronto.edu/~fritz/absps/momentum.pdf}{On the
  importance of initialization and momentum in deep learning}
\end{itemize}



\subsubsection{TFOptimizer  {\href{https://github.com/keras-team/keras/blob/master/keras/optimizers.py\#L649}{{[}source{]}}}}

\begin{Shaded}
\begin{Highlighting}[]
\NormalTok{keras.optimizers.TFOptimizer(optimizer)}
\end{Highlighting}
\end{Shaded}

原生Tensorlfow优化器的包装类(wrapper class)。
\newpage
