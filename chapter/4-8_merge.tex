
\subsection{融合层 Merge}
\subsubsection{Add {\href{https://github.com/keras-team/keras/blob/master/keras/layers/merge.py\#L193}{{[}source{]}}}}

\begin{Shaded}
\begin{Highlighting}[]
\NormalTok{keras.layers.Add()}
\end{Highlighting}
\end{Shaded}

计算一个列表的输入张量的和。

相加层接受一个列表的张量, 所有的张量必须有相同的输入尺寸,
然后返回一个张量(和输入张量尺寸相同)。

\textbf{例子}

\begin{Shaded}
\begin{Highlighting}[]
\ImportTok{import} \NormalTok{keras}

\NormalTok{input1 }\OperatorTok{=} \NormalTok{keras.layers.Input(shape}\OperatorTok{=}\NormalTok{(}\DecValTok{16}\NormalTok{,))}
\NormalTok{x1 }\OperatorTok{=} \NormalTok{keras.layers.Dense(}\DecValTok{8}\NormalTok{, activation}\OperatorTok{=}\StringTok{'relu'}\NormalTok{)(input1)}
\NormalTok{input2 }\OperatorTok{=} \NormalTok{keras.layers.Input(shape}\OperatorTok{=}\NormalTok{(}\DecValTok{32}\NormalTok{,))}
\NormalTok{x2 }\OperatorTok{=} \NormalTok{keras.layers.Dense(}\DecValTok{8}\NormalTok{, activation}\OperatorTok{=}\StringTok{'relu'}\NormalTok{)(input2)}
\NormalTok{added }\OperatorTok{=} \NormalTok{keras.layers.Add()([x1, x2])  }\CommentTok{# 相当于 added = keras.layers.add([x1, x2])}

\NormalTok{out }\OperatorTok{=} \NormalTok{keras.layers.Dense(}\DecValTok{4}\NormalTok{)(added)}
\NormalTok{model }\OperatorTok{=} \NormalTok{keras.models.Model(inputs}\OperatorTok{=}\NormalTok{[input1, input2], outputs}\OperatorTok{=}\NormalTok{out)}
\end{Highlighting}
\end{Shaded}




\subsubsection{Subtract {\href{https://github.com/keras-team/keras/blob/master/keras/layers/merge.py\#L223}{{[}source{]}}}}

\begin{Shaded}
\begin{Highlighting}[]
\NormalTok{keras.layers.Subtract()}
\end{Highlighting}
\end{Shaded}

计算两个输入张量的差。

相减层接受一个长度为 2 的张量列表,
两个张量必须有相同的尺寸,然后返回一个值为 (inputs{[}0{]} -
inputs{[}1{]}) 的张量, 输出张量和输入张量尺寸相同。

\textbf{例子}

\begin{Shaded}
\begin{Highlighting}[]
\ImportTok{import} \NormalTok{keras}

\NormalTok{input1 }\OperatorTok{=} \NormalTok{keras.layers.Input(shape}\OperatorTok{=}\NormalTok{(}\DecValTok{16}\NormalTok{,))}
\NormalTok{x1 }\OperatorTok{=} \NormalTok{keras.layers.Dense(}\DecValTok{8}\NormalTok{, activation}\OperatorTok{=}\StringTok{'relu'}\NormalTok{)(input1)}
\NormalTok{input2 }\OperatorTok{=} \NormalTok{keras.layers.Input(shape}\OperatorTok{=}\NormalTok{(}\DecValTok{32}\NormalTok{,))}
\NormalTok{x2 }\OperatorTok{=} \NormalTok{keras.layers.Dense(}\DecValTok{8}\NormalTok{, activation}\OperatorTok{=}\StringTok{'relu'}\NormalTok{)(input2)}
\CommentTok{# 相当于 subtracted = keras.layers.subtract([x1, x2])}
\NormalTok{subtracted }\OperatorTok{=} \NormalTok{keras.layers.Subtract()([x1, x2])}

\NormalTok{out }\OperatorTok{=} \NormalTok{keras.layers.Dense(}\DecValTok{4}\NormalTok{)(subtracted)}
\NormalTok{model }\OperatorTok{=} \NormalTok{keras.models.Model(inputs}\OperatorTok{=}\NormalTok{[input1, input2], outputs}\OperatorTok{=}\NormalTok{out)}
\end{Highlighting}
\end{Shaded}




\subsubsection{Multiply {\href{https://github.com/keras-team/keras/blob/master/keras/layers/merge.py\#L260}{{[}source{]}}}}

\begin{Shaded}
\begin{Highlighting}[]
\NormalTok{keras.layers.Multiply()}
\end{Highlighting}
\end{Shaded}

计算一个列表的输入张量的(逐元素间的)乘积。

相乘层接受一个列表的张量, 所有的张量必须有相同的输入尺寸,
然后返回一个张量(和输入张量尺寸相同)。




\subsubsection{Average {\href{https://github.com/keras-team/keras/blob/master/keras/layers/merge.py\#L275}{{[}source{]}}}}

\begin{Shaded}
\begin{Highlighting}[]
\NormalTok{keras.layers.Average()}
\end{Highlighting}
\end{Shaded}

计算一个列表的输入张量的平均值。

平均层接受一个列表的张量, 所有的张量必须有相同的输入尺寸,
然后返回一个张量(和输入张量尺寸相同)。




\subsubsection{Maximum {\href{https://github.com/keras-team/keras/blob/master/keras/layers/merge.py\#L290}{{[}source{]}}}}

\begin{Shaded}
\begin{Highlighting}[]
\NormalTok{keras.layers.Maximum()}
\end{Highlighting}
\end{Shaded}

计算一个列表的输入张量的(逐元素间的)最大值。

最大层接受一个列表的张量, 所有的张量必须有相同的输入尺寸,
然后返回一个张量(和输入张量尺寸相同)。




\subsubsection{Concatenate {\href{https://github.com/keras-team/keras/blob/master/keras/layers/merge.py\#L320}{{[}source{]}}}}

\begin{Shaded}
\begin{Highlighting}[]
\NormalTok{keras.layers.Concatenate(axis}\OperatorTok{=-}\DecValTok{1}\NormalTok{)}
\end{Highlighting}
\end{Shaded}

串联一个列表的输入张量。

串联层接受一个列表的张量, 除了串联轴之外,其他的尺寸都必须相同,
然后返回一个由所有输入张量串联起来的输出张量。

\textbf{参数}

\begin{itemize}
\tightlist
\item
  \textbf{axis}: 串联的轴。
\item
  \_\_**kwargs\_\_: 层关键字参数。
\end{itemize}




\subsubsection{Dot {\href{https://github.com/keras-team/keras/blob/master/keras/layers/merge.py\#L408}{{[}source{]}}}}

\begin{Shaded}
\begin{Highlighting}[]
\NormalTok{keras.layers.Dot(axes, normalize}\OperatorTok{=}\VariableTok{False}\NormalTok{)}
\end{Highlighting}
\end{Shaded}

计算两个张量之间样本的点积。

例如,如果作用于输入尺寸为 \texttt{(batch\_size,\ n)} 的两个张量
\texttt{a} 和 \texttt{b}, 那么输出结果就会是尺寸为
\texttt{(batch\_size,\ 1)} 的一个张量。 在这个张量中,每一个条目
\texttt{i} 是 \texttt{a{[}i{]}} 和 \texttt{b{[}i{]}} 之间的点积。

\textbf{参数}

\begin{itemize}
\tightlist
\item
  \textbf{axes}: 整数或者整数元组, 一个或者几个进行点积的轴。
\item
  \textbf{normalize}: 是否在点积之前对即将进行点积的轴进行 L2 标准化。
  如果设置成 True,那么输出两个样本之间的余弦相似值。
\item
  \_\_**kwargs\_\_: 层关键字参数。
\end{itemize}


\subsubsection{add}\label{add}

\begin{Shaded}
\begin{Highlighting}[]
\NormalTok{keras.layers.add(inputs)}
\end{Highlighting}
\end{Shaded}

\texttt{Add} 层的函数式接口。

\textbf{参数}

\begin{itemize}
\tightlist
\item
  \textbf{inputs}: 一个列表的输入张量(列表大小至少为 2)。
\item
  \_\_**kwargs\_\_: 层关键字参数。
\end{itemize}

\textbf{返回}

一个张量,所有输入张量的和。

\textbf{例子}

\begin{Shaded}
\begin{Highlighting}[]
\ImportTok{import} \NormalTok{keras}

\NormalTok{input1 }\OperatorTok{=} \NormalTok{keras.layers.Input(shape}\OperatorTok{=}\NormalTok{(}\DecValTok{16}\NormalTok{,))}
\NormalTok{x1 }\OperatorTok{=} \NormalTok{keras.layers.Dense(}\DecValTok{8}\NormalTok{, activation}\OperatorTok{=}\StringTok{'relu'}\NormalTok{)(input1)}
\NormalTok{input2 }\OperatorTok{=} \NormalTok{keras.layers.Input(shape}\OperatorTok{=}\NormalTok{(}\DecValTok{32}\NormalTok{,))}
\NormalTok{x2 }\OperatorTok{=} \NormalTok{keras.layers.Dense(}\DecValTok{8}\NormalTok{, activation}\OperatorTok{=}\StringTok{'relu'}\NormalTok{)(input2)}
\NormalTok{added }\OperatorTok{=} \NormalTok{keras.layers.add([x1, x2])}

\NormalTok{out }\OperatorTok{=} \NormalTok{keras.layers.Dense(}\DecValTok{4}\NormalTok{)(added)}
\NormalTok{model }\OperatorTok{=} \NormalTok{keras.models.Model(inputs}\OperatorTok{=}\NormalTok{[input1, input2], outputs}\OperatorTok{=}\NormalTok{out)}
\end{Highlighting}
\end{Shaded}



\subsubsection{subtract}\label{subtract}

\begin{Shaded}
\begin{Highlighting}[]
\NormalTok{keras.layers.subtract(inputs)}
\end{Highlighting}
\end{Shaded}

\texttt{Subtract} 层的函数式接口。

\textbf{参数}

\begin{itemize}
\tightlist
\item
  \textbf{inputs}: 一个列表的输入张量(列表大小准确为 2)。
\item
  \_\_**kwargs\_\_: 层的关键字参数。
\end{itemize}

\textbf{返回}

一个张量,两个输入张量的差。

\textbf{例子}

\begin{Shaded}
\begin{Highlighting}[]
\ImportTok{import} \NormalTok{keras}

\NormalTok{input1 }\OperatorTok{=} \NormalTok{keras.layers.Input(shape}\OperatorTok{=}\NormalTok{(}\DecValTok{16}\NormalTok{,))}
\NormalTok{x1 }\OperatorTok{=} \NormalTok{keras.layers.Dense(}\DecValTok{8}\NormalTok{, activation}\OperatorTok{=}\StringTok{'relu'}\NormalTok{)(input1)}
\NormalTok{input2 }\OperatorTok{=} \NormalTok{keras.layers.Input(shape}\OperatorTok{=}\NormalTok{(}\DecValTok{32}\NormalTok{,))}
\NormalTok{x2 }\OperatorTok{=} \NormalTok{keras.layers.Dense(}\DecValTok{8}\NormalTok{, activation}\OperatorTok{=}\StringTok{'relu'}\NormalTok{)(input2)}
\NormalTok{subtracted }\OperatorTok{=} \NormalTok{keras.layers.subtract([x1, x2])}

\NormalTok{out }\OperatorTok{=} \NormalTok{keras.layers.Dense(}\DecValTok{4}\NormalTok{)(subtracted)}
\NormalTok{model }\OperatorTok{=} \NormalTok{keras.models.Model(inputs}\OperatorTok{=}\NormalTok{[input1, input2], outputs}\OperatorTok{=}\NormalTok{out)}
\end{Highlighting}
\end{Shaded}



\subsubsection{multiply}\label{multiply}

\begin{Shaded}
\begin{Highlighting}[]
\NormalTok{keras.layers.multiply(inputs)}
\end{Highlighting}
\end{Shaded}

\texttt{Multiply} 层的函数式接口。

\textbf{参数}

\begin{itemize}
\tightlist
\item
  \textbf{inputs}: 一个列表的输入张量(列表大小至少为 2)。
\item
  \_\_**kwargs\_\_: 层的关键字参数。
\end{itemize}

\textbf{返回}

一个张量,所有输入张量的逐元素乘积。



\subsubsection{average}\label{average}

\begin{Shaded}
\begin{Highlighting}[]
\NormalTok{keras.layers.average(inputs)}
\end{Highlighting}
\end{Shaded}

\texttt{Average} 层的函数式接口。

\textbf{参数}

\begin{itemize}
\tightlist
\item
  \textbf{inputs}: 一个列表的输入张量(列表大小至少为 2)。
\item
  \_\_**kwargs\_\_: 层的关键字参数。
\end{itemize}

\textbf{返回}

一个张量,所有输入张量的平均值。



\subsubsection{maximum}\label{maximum}

\begin{Shaded}
\begin{Highlighting}[]
\NormalTok{keras.layers.maximum(inputs)}
\end{Highlighting}
\end{Shaded}

\texttt{Maximum} 层的函数式接口。

\textbf{参数}

\begin{itemize}
\tightlist
\item
  \textbf{inputs}: 一个列表的输入张量(列表大小至少为 2)。
\item
  \_\_**kwargs\_\_: 层的关键字参数。
\end{itemize}

\textbf{返回}

一个张量,所有张量的逐元素的最大值。



\subsubsection{concatenate}\label{concatenate}

\begin{Shaded}
\begin{Highlighting}[]
\NormalTok{keras.layers.concatenate(inputs, axis}\OperatorTok{=-}\DecValTok{1}\NormalTok{)}
\end{Highlighting}
\end{Shaded}

\texttt{Concatenate} 层的函数式接口。

\textbf{参数}

\begin{itemize}
\tightlist
\item
  \textbf{inputs}: 一个列表的输入张量(列表大小至少为 2)。
\item
  \textbf{axis}: 串联的轴。
\item
  \_\_**kwargs\_\_: 层的关键字参数。
\end{itemize}

\textbf{返回}

一个张量,所有输入张量通过 \texttt{axis} 轴串联起来的输出张量。



\subsubsection{dot}\label{dot}

\begin{Shaded}
\begin{Highlighting}[]
\NormalTok{keras.layers.dot(inputs, axes, normalize}\OperatorTok{=}\VariableTok{False}\NormalTok{)}
\end{Highlighting}
\end{Shaded}

\texttt{Dot} 层的函数式接口。

\textbf{参数}

\begin{itemize}
\tightlist
\item
  \textbf{inputs}: 一个列表的输入张量(列表大小至少为 2)。
\item
  \textbf{axes}: 整数或者整数元组, 一个或者几个进行点积的轴。
\item
  \textbf{normalize}: 是否在点积之前对即将进行点积的轴进行 L2 标准化。
  如果设置成 True,那么输出两个样本之间的余弦相似值。
\item
  \_\_**kwargs\_\_: 层的关键字参数。
\end{itemize}

\textbf{返回}

一个张量,所有输入张量样本之间的点积。

\newpage
